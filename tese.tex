\documentclass[defaultstyle,10pt,master,Helvetica]{01.thesis}

\usepackage[utf8]{inputenc}
\usepackage[T1]{fontenc}
\usepackage{amsmath,amssymb,amsfonts}
\usepackage{graphicx}
\usepackage{textcomp}
\usepackage{booktabs}
\usepackage{framed}
\usepackage{float}
\usepackage{xcolor}
\usepackage{footnote}
\usepackage{xfrac}
\usepackage{balance}
\usepackage{array}
\usepackage{changepage}
\usepackage{listings}
\usepackage{wrapfig}
\usepackage{amsmath}
\usepackage{graphicx}
\usepackage{textcomp}
\usepackage{booktabs}
\usepackage{caption}
\usepackage{subcaption}
\usepackage[inline]{enumitem}
\usepackage{xcolor}
\usepackage{color, colortbl}
\usepackage{multicol} 
\usepackage{CJKutf8}



\definecolor{main}{HTML}{5989cf}    % setting main color to be used
\definecolor{sub}{HTML}{cde4ff}     % setting sub color to be used

                    % setting global options for tcolorbox

\definecolor{Gray}{gray}{0.9}
\usepackage{soul}

\sethlcolor{yellow}

%% general
\newcommand{\CodeIn}[1]{\begin{small}\texttt{#1}\end{small}}
\newcommand{\Fix}[1]{\textbf{\color{red}[#1]}}
\newcommand{\Ignore}[1]{}
\newcommand{\etal}{et al.}
\newcommand{\myeg}{e.g.}
\newcommand{\ie}{i.e.}
\newcommand{\aka}{a.k.a.}
\newcommand{\etc}{etc.}
\newcommand{\rahman}{Rahman \etal}

\DeclareRobustCommand{\mhl}[1]{%
  \text{\hl{$#1$}}%
}

%% numbers
\newcommand{\noRulesSlic}{$7$}
\newcommand{\slicPrecision}{$99\%$}
\newcommand{\totalMinedRepos}{$1419$}
\newcommand{\totalMinedScripts}{$34574$}
\newcommand{\fpSampleSize}{$502$}
\newcommand{\proportionalSampleSize}{$250$}
\newcommand{\uniformSampleSize}{$252$}
\newcommand{\casesPerWarning}{$36$}

%%%%%%%%%%%%%%%%%%%%%%%%%%%%%%%%%%%%%%%%
%%%           SLIC RESULTS           %%%
%%%%%%%%%%%%%%%%%%%%%%%%%%%%%%%%%%%%%%%%

\newcommand{\hardcodedSecretsMined}{$22365$}
\newcommand{\httpWithoutTLSMined}{$3757$}
\newcommand{\suspiciousCommentsMined}{$2780$}
\newcommand{\weakCryptoMined}{$1489$}
\newcommand{\emptyPassMined}{$769$}
\newcommand{\invalidIPMined}{$684$}
\newcommand{\adminDefaultMined}{$146$}

\newcommand{\hardcodedSecretsProportional}{$174$}
\newcommand{\httpWithoutTLSProportional}{$29$}
\newcommand{\suspiciousCommentsProportional}{$22$}
\newcommand{\weakCryptoProportional}{$11$}
\newcommand{\emptyPassProportional}{$6$}
\newcommand{\invalidIPProportional}{$6$}
\newcommand{\adminDefaultProportional}{$2$}

\newcommand{\fpHardcodedSecretsProportional}{$52$}
\newcommand{\fpHttpWithoutTLSProportional}{$20$}
\newcommand{\fpSuspiciousCommentsProportional}{$12$}
\newcommand{\fpWeakCryptoProportional}{$4$}
\newcommand{\fpInvalidIPProportional}{$0$}
\newcommand{\fpEmptyPassProportional}{$2$}
\newcommand{\fpAdminDefaultProportional}{$1$}
\newcommand{\fpProportionalSample}{$91$}

\newcommand{\tpHardcodedSecretsProportional}{$122$}
\newcommand{\tpHttpWithoutTLSProportional}{$9$}
\newcommand{\tpSuspiciousCommentsProportional}{$10$}
\newcommand{\tpWeakCryptoProportional}{$7$}
\newcommand{\tpInvalidIPProportional}{$6$}
\newcommand{\tpEmptyPassProportional}{$4$}
\newcommand{\tpAdminDefaultProportional}{$1$}
\newcommand{\tpProportionalSample}{$159$}

\newcommand{\precHardcodedSecretsProportional}{$0.70$}
\newcommand{\precHttpWithoutTLSProportional}{$0.31$}
\newcommand{\precSuspiciousCommentsProportional}{$0.45$}
\newcommand{\precWeakCryptoProportional}{$0.64$}
\newcommand{\precInvalidIPProportional}{$1.00$}
\newcommand{\precEmptyPassProportional}{$0.67$}
\newcommand{\precAdminDefaultProportional}{$0.50$}
\newcommand{\precTotalProportional}{$0.64$}

\newcommand{\fpHardcodedSecretsUniform}{$10$}
\newcommand{\fpHttpWithoutTLSUniform}{$26$}
\newcommand{\fpSuspiciousCommentsUniform}{$28$}
\newcommand{\fpWeakCryptoUniform}{$11$}
\newcommand{\fpInvalidIPUniform}{$8$}
\newcommand{\fpEmptyPassUniform}{$15$}
\newcommand{\fpAdminDefaultUniform}{$15$}
\newcommand{\fpUniformSample}{$113$}

\newcommand{\tpHardcodedSecretsUniform}{$26$}
\newcommand{\tpHttpWithoutTLSUniform}{$10$}
\newcommand{\tpSuspiciousCommentsUniform}{$8$}
\newcommand{\tpWeakCryptoUniform}{$25$}
\newcommand{\tpInvalidIPUniform}{$28$}
\newcommand{\tpEmptyPassUniform}{$21$}
\newcommand{\tpAdminDefaultUniform}{$21$}
\newcommand{\tpUniformSample}{$139$}

\newcommand{\precHardcodedSecretsUniform}{$0.72$}
\newcommand{\precHttpWithoutTLSUniform}{$0.28$}
\newcommand{\precSuspiciousCommentsUniform}{$0.22$}
\newcommand{\precWeakCryptoUniform}{$0.69$}
\newcommand{\precInvalidIPUniform}{$0.78$}
\newcommand{\precEmptyPassUniform}{$0.58$}
\newcommand{\precAdminDefaultUniform}{$0.58$}
\newcommand{\precTotalUniform}{$0.55$}

%%%%%%%%%%%%%%%%%%%%%%%%%%%%%%%%%%%%%%%%
%%%   InfraSecure v0.1.0 RESULTS     %%%
%%%%%%%%%%%%%%%%%%%%%%%%%%%%%%%%%%%%%%%%

\newcommand{\fpHardcodedSecretsInfraSecureProportional}{$22$}
\newcommand{\fpHttpWithoutTLSInfraSecureProportional}{$17$}
\newcommand{\fpSuspiciousCommentsInfraSecureProportional}{$2$}
\newcommand{\fpWeakCryptoInfraSecureProportional}{$2$}
\newcommand{\fpInvalidIPInfraSecureProportional}{$0$}
\newcommand{\fpEmptyPassInfraSecureProportional}{$2$}
\newcommand{\fpAdminDefaultInfraSecureProportional}{$1$}
\newcommand{\fpInfraSecureProportionalSample}{$46$}

\newcommand{\tpHardcodedSecretsInfraSecureProportional}{$118$}
\newcommand{\tpHttpWithoutTLSInfraSecureProportional}{$8$}
\newcommand{\tpSuspiciousCommentsInfraSecureProportional}{$5$}
\newcommand{\tpWeakCryptoInfraSecureProportional}{$5$}
\newcommand{\tpInvalidIPInfraSecureProportional}{$6$}
\newcommand{\tpEmptyPassInfraSecureProportional}{$4$}
\newcommand{\tpAdminDefaultInfraSecureProportional}{$1$}
\newcommand{\tpInfraSecureProportionalSample}{$147$}

\newcommand{\precHardcodedSecretsInfraSecureProportional}{$0.84$}
\newcommand{\precHttpWithoutTLSInfraSecureProportional}{$0.32$}
\newcommand{\precSuspiciousCommentsInfraSecureProportional}{$0.71$}
\newcommand{\precWeakCryptoInfraSecureProportional}{$0.71$}
\newcommand{\precInvalidIPInfraSecureProportional}{$1.00$}
\newcommand{\precEmptyPassInfraSecureProportional}{$0.67$}
\newcommand{\precAdminDefaultInfraSecureProportional}{$0.50$}
\newcommand{\precTotalInfraSecureProportional}{$0.76$}

\newcommand{\fpHardcodedSecretsInfraSecureUniform}{$4$}
\newcommand{\fpHttpWithoutTLSInfraSecureUniform}{$23$}
\newcommand{\fpSuspiciousCommentsInfraSecureUniform}{$10$}
\newcommand{\fpWeakCryptoInfraSecureUniform}{$2$}
\newcommand{\fpInvalidIPInfraSecureUniform}{$1$}
\newcommand{\fpEmptyPassInfraSecureUniform}{$15$}
\newcommand{\fpAdminDefaultInfraSecureUniform}{$15$}
\newcommand{\fpInfraSecureUniformSample}{$70$}

\newcommand{\tpHardcodedSecretsInfraSecureUniform}{$24$}
\newcommand{\tpHttpWithoutTLSInfraSecureUniform}{$9$}
\newcommand{\tpSuspiciousCommentsInfraSecureUniform}{$6$}
\newcommand{\tpWeakCryptoInfraSecureUniform}{$23$}
\newcommand{\tpInvalidIPInfraSecureUniform}{$28$}
\newcommand{\tpEmptyPassInfraSecureUniform}{$21$}
\newcommand{\tpAdminDefaultInfraSecureUniform}{$20$}
\newcommand{\tpInfraSecureUniformSample}{$131$}

\newcommand{\precHardcodedSecretsInfraSecureUniform}{$0.86$}
\newcommand{\precHttpWithoutTLSInfraSecureUniform}{$0.28$}
\newcommand{\precSuspiciousCommentsInfraSecureUniform}{$0.38$}
\newcommand{\precWeakCryptoInfraSecureUniform}{$0.92$}
\newcommand{\precInvalidIPInfraSecureUniform}{$0.97$}
\newcommand{\precEmptyPassInfraSecureUniform}{$0.58$}
\newcommand{\precAdminDefaultInfraSecureUniform}{$0.57$}
\newcommand{\precTotalInfraSecureUniform}{$0.65$}

%%%%%%%%%%%%%%%%%%%%%%%%%%%%%%%%%%%%%%%%




\newcommand{\preliminarPrecision}{$64\%$}


\newcommand{\botTotalScripts}{$3740$}
\newcommand{\botTotalRepos}{$86$}
\newcommand{\botWarningsRepos}{$287$}
\newcommand{\botTotalWarnings}{$1975$}
\newcommand{\botScriptsWarnings}{$1147$}
\newcommand{\botTotalIssues}{$228$}
\newcommand{\botTotalIssuesAnswers}{$51$}
\newcommand{\botTotalWarningsAnswers}{$298$}
\newcommand{\botIssues}{$226$}
\newcommand{\botFinalIssues}{$33$}
\newcommand{\botPrecision}{$28\%$}



\newcommand{\hardcodedSecretsBot}{$1917$}
\newcommand{\httpWithoutTLSBot}{$353$}
\newcommand{\suspiciousCommentsBot}{$342$}
\newcommand{\weakCryptoBot}{$103$}
\newcommand{\emptyPassBot}{$102$}
\newcommand{\invalidIPBot}{$93$}
\newcommand{\adminDefaultBot}{$29$}

\newcommand{\fpHardcodedSecretsBot}{$x$}
\newcommand{\fpHttpWithoutTLSBot}{$x$}
\newcommand{\fpSuspiciousCommentsBot}{$x$}
\newcommand{\fpWeakCryptoBot}{$x$}
\newcommand{\fpEmptyPassBot}{$x$}
\newcommand{\fpInvalidIPBot}{$x$}
\newcommand{\fpAdminDefaultBot}{$x$}

\newcommand{\tpHardcodedSecretsBot}{$x$}
\newcommand{\tpHttpWithoutTLSBot}{$x$}
\newcommand{\tpSuspiciousCommentsBot}{$x$}
\newcommand{\tpWeakCryptoBot}{$x$}
\newcommand{\tpEmptyPassBot}{$x$}
\newcommand{\tpInvalidIPBot}{$x$}
\newcommand{\tpAdminDefaultBot}{$x$}

\newcommand{\hardcodedSecretsAnswered}{$647$}
\newcommand{\httpWithoutTLSAnswered}{$188$}
\newcommand{\suspiciousCommentsAnswered}{$163$}
\newcommand{\weakCryptoAnswered}{$12$}
\newcommand{\emptyPassAnswered}{$46$}
\newcommand{\invalidIPAnswered}{$14$}
\newcommand{\adminDefaultAnswered}{$4$}
\newcommand{\botTotalAnswered}{$298$}

\newcommand{\minedReposWarnings}{$1093$}
\newcommand{\minedScriptsWarnings}{$9144$}
\newcommand{\minedWarnings}{$31990$}
\newcommand{\akondFpTotalScripts}{$140$}

\newcommand{\akondFpTotalRepos}{$74$}
\newcommand{\akondFpScriptsWarnings}{$27$}
\newcommand{\akondFpScriptsWarningsPercentage}{$19.3\%$}
\newcommand{\akondFpWarnings}{$58$}
\newcommand{\akondPrecisionRecall}{$0.99$}

%% Study with the practicioners

\newcommand{\newRules}{$3$}
\newcommand{\noProfessionals}{$131$}
\newcommand{\noProfessionalsProlificRaw}{$431$}
\newcommand{\noProfessionalsProlificCleaned}{$227$}
\newcommand{\noProfessionalsCommunity}{$14$}
\newcommand{\noProfessionalsProlific}{$117$}
\newcommand{\noWarningsPerPracticioner}{$3$}
\newcommand{\finalPrecision}{$83\%$}
\newcommand{\totalWarningsPrac}{$339$}
 
%% names
\newcommand{\slic}{\textsc{SLIC}}
\newcommand{\puplint}{\texttt{puppet-lint}}
\newcommand{\toolname}{\textsc{InfraSecure}}
\newcommand{\iac}{IaC}
\newcommand{\github}{GitHub}
% other paragraphs
\newcommand{\Space}[1]{}
\newcommand{\Contrib}[1]{$\star$#1}

\newcommand{\repPackage}{\url{https://figshare.com/s/6b6a769b1393eae0774c}}


\newcolumntype{L}[1]{>{\raggedright\let\newline\\\arraybackslash\hspace{0pt}}m{#1}}

\definecolor{mygreen}{rgb}{0,0.6,0}
\definecolor{lightgreen}{rgb}{0.6,0.9,0.6}
\definecolor{lightyellow}{rgb}{0.9,0.9,0.6}
\definecolor{lightorange}{rgb}{0.9,0.8,0.6}
\definecolor{lightred}{rgb}{0.9,0.7,0.7}
\definecolor{mygray}{rgb}{0.5,0.5,0.5}
\definecolor{lightgray}{rgb}{0.8,0.8,0.8}
\definecolor{mymauve}{rgb}{0.58,0,0.82}

\usepackage{pifont}
\newcommand{\lstbg}[3][0pt]{{\fboxsep#1\colorbox{#2}{\strut #3}}}

\lstdefinestyle{CStyle} {
    language=C,
    backgroundcolor=\color{white},   % choose the background color
    basicstyle=\ttfamily\scriptsize,        % size of fonts used for the code
    breaklines=true,                 % automatic line breaking only at whitespace
    captionpos=b,                    % sets the caption-position to bottom
    commentstyle=\color{mygray}\bfseries,    % comment style
    escapeinside={\%*}{*)},          % if you want to add LaTeX within your code
    keywordstyle=\color{blue},       % keyword style
    stringstyle=\color{mymauve},     % string literal style
    frame=single,
    numbers=left,
    stepnumber=1,
    xleftmargin=2em,
    escapeinside={/*!}{!*/},
    moredelim=**[l][\color{mygreen}]{+\ },
    moredelim=*[l][\color{red}]{-\ }
}

\lstdefinestyle{JavaStyle} {
    language=Java,
    backgroundcolor=\color{white},   % choose the background color
    basicstyle=\ttfamily\scriptsize,        % size of fonts used for the code
    breaklines=true,                 % automatic line breaking only at whitespace
    captionpos=b,                    % sets the caption-position to bottom
    commentstyle=\color{mygray}\bfseries,    % comment style
    escapeinside={\%*}{*)},          % if you want to add LaTeX within your code
    keywordstyle=\color{blue},       % keyword style
    stringstyle=\color{mymauve},     % string literal style
    frame=single,
    numbers=left,
    stepnumber=1,
    xleftmargin=2em,
    escapeinside={/*!}{!*/},
    moredelim=**[l][\color{mygreen}]{+\ },
    moredelim=*[l][\color{red}]{-\ }
}

\lstdefinestyle{PHPStyle} {
    language=PHP,
    alsolanguage=HTML,
    backgroundcolor=\color{white},   % choose the background color
    basicstyle=\ttfamily\scriptsize,        % size of fonts used for the code
    breaklines=true,                 % automatic line breaking only at whitespace
    captionpos=b,                    % sets the caption-position to bottom
    commentstyle=\color{mygray}\bfseries,    % comment style
    escapeinside={\%*}{*)},          % if you want to add LaTeX within your code
    keywordstyle=\color{blue},       % keyword style
    stringstyle=\color{mymauve},     % string literal style
    frame=single,
	numbers=left,
    stepnumber=1,
	xleftmargin=2em,
    escapeinside={/*!}{!*/},
    moredelim=**[l][\color{mygreen}]{+\ },
    moredelim=*[l][\color{red}]{-\ }
}

\newcounter{lstannotation}
\setcounter{lstannotation}{0}
\renewcommand{\thelstannotation}{\ding{\number\numexpr181+\arabic{lstannotation}}}
\newcommand{\annotation}[1]{\refstepcounter{lstannotation}\label{#1}\thelstannotation}

\newboolean{showcomments}
\setboolean{showcomments}{true}
%\setboolean{showcomments}{false}

\ifthenelse{\boolean{showcomments}}
  {\newcommand{\nb}[3]{
  {\color{#2}\small\fbox{\bfseries\sffamily\scriptsize#1}}
  {\color{#2}\sffamily\small$\triangleright~$\textit{\small #3}$~\triangleleft$}
  }
  }
  {\newcommand{\nb}[3]{}
  }

\newcommand\Sofia[1]{\nb{Sofia}{red}{#1}}
\newcommand\Luis[1]{\nb{Luis}{mygreen}{#1}}
\newcommand\Rui[1]{\nb{Rui}{blue}{#1}}

\makeatletter
\newcommand\footnoteref[1]{\protected@xdef\@thefnmark{\ref{#1}}\@footnotemark}
\makeatother

%% Packages
\input{00.Definitions/packages.tex}
%% Page formatting
\input{00.Definitions/pagesetup.tex}


%-----------------------------------------------------------
%-----------------------------------------------------------
\begin{document}
%% Use Main document Language
\selectlanguage{english}
%-- ! --> Usar 'portuguese' para versão portuguesa 
%% ------
\pagestyle{begin}

%% ----------- COVER --------------
% After deciding the language in which you'll write the thesis -- PT (Portuguese) or EN (English) -- you can
% edit and select (uncomment) the respective cover file.
%%%---------DRAFT
\setcounter{page}{1} \pagenumbering{Alph}

% Add PDF bookmark 
\pdfbookmark[0]{Title}{Title}

%%% LOGO
\thispagestyle{empty}
\begin{flushleft} ~\\ \vspace{-12mm} \hspace{-12mm}  \includegraphics[width=50mm]{Cover/istlogo} 
 
 %%% Instituição
\centering
\LARGE \textbf{UNIVERSIDADE DE LISBOA \\ INSTITUTO SUPERIOR TÉCNICO}
%%% espaço sem gráficos
\vspace{30mm}

%%% Optional Image
%\vspace{10mm}
%~\\ \vspace{50mm} % gráficos
%\\ \begin{center} \includegraphics[height=50mm]{Cover/coverimage}  \end{center} % gráficos
 \vspace{5mm}
 
 %%% Titulo
\centering
\LARGE \textbf{Static Aplication Security Testing Tools}
\\ \vspace{10mm}
\Large SCOPE, IMPROVEMENTS AND FUTURE APPLICATIONS
\\ \vspace{15mm}
%\\ \vspace{25mm}  % NO SUBTITLE
\Large \textbf{Sofia Oliveira Reis} \\
\vspace{4cm}

\begin{minipage}{\textwidth}
\begin{tabularx}{\textwidth}{ l }
\large \textbf{Supervisor} : Rui Maranhão Abreu\\
 \large \textbf{Co-Supervisor} :  João Ferreira\\
\end{tabularx}

\end{minipage}
%
\\ \vspace{27mm}
%\vspace{12mm}
\centering
\large \textbf{Thesis specifically prepared to obtain the PhD Degree in}\\
\large Computer Science and Engineering\\
%\\ \vspace{2mm}
\vspace{18mm}
\Large \textbf{Draft}
 
\vspace{15mm}

\large \textbf{\todaythesis\today} \\
% \large \textbf{December 2017} \\
\let\thepage\relax
\end{flushleft}
\pagebreak
 
%\input{0.Inicio/1.cover_PT_draft.tex}
%%%----------FINAL --- uncomment all code bellow and comment code above.
%\input{0.Inicio/1.cover_EN_final.tex} 
%%\input{0.Inicio/1.cover_PT_final.tex} 
%\clearpage
%\thispagestyle{empty}
%\cleardoublepage
%\input{0.Inicio/1.1stpage_EN_final.tex} 
%%\input{0.Inicio/1.1stpage_PT_final.tex} 
%%\clearpage


%%-------------------------------------
\clearpage
% Since I am using double sided pages, the second page should be white.
% Remember that when delivering the dissertation, IST requires for the cover to appear twice.

\thispagestyle{empty}
\cleardoublepage

\setcounter{page}{1} \pagenumbering{roman} % --- Start with Roman numbering ---

\baselineskip 18pt % line spacing: -12pt for single spacing
                   %               -18pt for 1 1/2 spacing
                   %               -24pt for double spacingnts}
                   
 %%%% Initial Chapters
 \selectlanguage{english}
\input{0.Inicio/4.Abstract.tex}
\input{0.Inicio/5.Keywords.tex}
\selectlanguage{portuguese}
\input{0.Inicio/4.Resumo.tex}
\input{0.Inicio/5.PalavrasChave.tex}
\input{0.Inicio/3.Acknowledgments.tex}
\input{0.Inicio/2.citation.tex} % --- Citation (optional) ---
%% Use Main document Language
\selectlanguage{english}
%-- ! --> Usar 'portuguese' para versão portuguesa.
%% ------
\input{0.Inicio/6.Tables.tex}
%\acresetall
%% Remain list of table titles are set manualy
\input{0.Inicio/7.Acronym.tex}
%\input{0.Inicio/8.symbols.tex} %changed to glossaries
\input{0.Inicio/8.Notation.tex}
%% Use Main document Language
\selectlanguage{english}
%-- ! --> Usar 'portuguese' para versão portuguesa.
%% Define the title of Chapter Table of Contents
\mtcsettitle{minitoc}{Contents}
%% ------
\pagestyle{documentsimple}%Simple head
% %%%%%%%%%%%%%%%%%%%%%%%%%%%%%%%%%%%%%%%%%%%%%%%%%%%%%%%%%%%%%%%%%%%%%%
% Dummy Chapter:
% %%%%%%%%%%%%%%%%%%%%%%%%%%%%%%%%%%%%%%%%%%%%%%%%%%%%%%%%%%%%%%%%%%%%%%

% %%%%%%%%%%%%%%%%%%%%%%%%%%%%%%%%%%%%%%%%%%%%%%%%%%%%%%%%%%%%%%%%%%%%%%
% The Introduction:
% %%%%%%%%%%%%%%%%%%%%%%%%%%%%%%%%%%%%%%%%%%%%%%%%%%%%%%%%%%%%%%%%%%%%%%
\fancychapter{Fixing Software Vulnerabilities Potentially Hinders Maintainability}
\label{cap:chapter}

\textit{Present the chapter content.}

\section{Section A}
\label{sec:sectiona}

\subsection{Subsection A}
\label{subsec:subasectionA}

This would be a citation \cite{dummy}.

The \gls{cop} defines the performance of the machine.
% The first time you use this, the acronym will be written in full with the acronym in parentheses: supernova (SN). At later times it will just print the acronym: SN.

Heat Pump's performance is given by the \gls{cophp}, a \gls{cop} for heat pumps.

\cleardoublepage

% %%%%%%%%%%%%%%%%%%%%%%%%%%%%%%%%%%%%%%%%%%%%%%%%%%%%%%%%%%%%%%%%%%%%%%
% Dummy Chapter:
% %%%%%%%%%%%%%%%%%%%%%%%%%%%%%%%%%%%%%%%%%%%%%%%%%%%%%%%%%%%%%%%%%%%%%%

% %%%%%%%%%%%%%%%%%%%%%%%%%%%%%%%%%%%%%%%%%%%%%%%%%%%%%%%%%%%%%%%%%%%%%%
% The Introduction:
% %%%%%%%%%%%%%%%%%%%%%%%%%%%%%%%%%%%%%%%%%%%%%%%%%%%%%%%%%%%%%%%%%%%%%%
\fancychapter{Static Application Security Testing: Scope and Opportunities}
\label{cap:chapter}

\textit{Catalog of academic and non-academic SAST tools.}

\section{Section A}
\label{sec:sectiona}

\subsection{Subsection A}
\label{subsec:subasectionA}

This would be a citation \cite{dummy}.

The \gls{cop} defines the performance of the machine.
% The first time you use this, the acronym will be written in full with the acronym in parentheses: supernova (SN). At later times it will just print the acronym: SN.

Heat Pump's performance is given by the \gls{cophp}, a \gls{cop} for heat pumps.

\section{Section B}
\label{sec:sectionb}

\subsection{Subsection A}
\label{subsec:subasectionB}


\cleardoublepage

% %%%%%%%%%%%%%%%%%%%%%%%%%%%%%%%%%%%%%%%%%%%%%%%%%%%%%%%%%%%%%%%%%%%%%%
% Dummy Chapter:
% %%%%%%%%%%%%%%%%%%%%%%%%%%%%%%%%%%%%%%%%%%%%%%%%%%%%%%%%%%%%%%%%%%%%%%

% %%%%%%%%%%%%%%%%%%%%%%%%%%%%%%%%%%%%%%%%%%%%%%%%%%%%%%%%%%%%%%%%%%%%%%
% The Introduction:
% %%%%%%%%%%%%%%%%%%%%%%%%%%%%%%%%%%%%%%%%%%%%%%%%%%%%%%%%%%%%%%%%%%%%%%
\fancychapter{SAST Testing and Validation}
\label{cap:chapter}

\textit{Present the chapter content.}

\section{Introduction}~\label{sec:intro}

% the problems of patch management 
% why do we need to patch 
% vulnerabilities faster -- EQUIFAX
Delays in deploying patches to known software security vulnerabilities 
have been the cause of major cybersecurity attacks~\cite{windows-cyberattack,failed-to-deploy-patch,hacker-news-patches,CISA-ALERT,non-applied-patches}. 
A practical example with major financial and reputation losses was the Equifax breach~\cite{failed-to-deploy-patch}: a failure to patch a 2-month-old critical bug in Apache Struts, which led to a sensitive data breach that impacted more than 143 million US consumers~\cite{EQUIFAX-1}. 
Timely patch management (i.e., the fast distribution and deployment 
of security fixes to users~\cite{SOFT-PATCH-MANAG-NIST,DBLP:conf/soups/LiRMMC19,10.5555/3488905.3488919,10.5555/3337432.3337437}) is one of the most effective and widely recognized strategies for 
% reducing the users' exposure window~\cite{DBLP:journals/corr/abs-2001-09148} and 
protecting
software systems against  cyberattacks~\cite{DISSANAYAKE2022106771,SOFT-PATCH-MANAG-NIST}. Yet, one important challenge still prevails, the \emph{lack of efficient patch triage systems} to identify and prioritize security patches~\cite{Zhang2021AnIO, SSPatcher2022,hacker-news-patches,DBLP:conf/soups/LiRMMC19}: current processes are largely manual (time-consuming) and prone to ignore important bug fixes such as the one behind Equifax~\cite{failed-to-deploy-patch}. An ideal and effective patch management process should include an effective audit system to identify patches~\cite{hacker-news-patches}. A recent study showed that $56.7\%$ of the commit messages attached to security patches are documented poorly and hinder  triage systems tasks (e.g., detection, prioritization, and assessment)~\cite{10.1145/3593434.3593481}.  

\textbf{Problem: Poor quality security commit messages hinder patch triage systems.} Previous work focused on using 
patch metadata (e.g., commit message) and code changes 
to explore automated patch detection~\cite{SSPatcher2022,reis2017secbench,9678720,DBLP:journals/corr/abs-1806-05893}, concluded that software vulnerability management (SVM) techniques could not rely purely on metadata to detect software vulnerabilities
due to the often inaccurate and incomplete  data~\cite{DBLP:journals/corr/abs-1806-05893}. In fact, only $38\%$ of commit messages used to ``silently'' patch software vulnerabilities in the past included security-related words~\cite{9678720}. Silent fixes
are performed when the vendor patches a vulnerability  without mentioning its existence anywhere (e.g., release notes, commit message, and more)~\cite{9678720}. This practice naturally leads to less informative patch documentation and hinders triage systems awareness and effectiveness. While this practice is usually used to protect vendors and systems, it
also limits the general knowledge pool of people who actually understand the vulnerability and know how to exploit it, which leaves users and defenders unprotected and unaware. According to the CERT Coordinated Vulnerability Disclosure (CVD) guide, silent patches should be avoided since knowing the existence of vulnerabilities and their patches is often the key driver to effective patch deployment~\cite{Householder2020}. 

%%%% Importance of the problem
% Providing well-structured and quality documentation of security patches can improve the  understanding of these software vulnerabilities for researchers and developers, which can ultimately enable faster patch management~\cite{Zhang2021AnIO,Householder2020,SSPatcher2022,10.1145/3593434.3593481}. Software vulnerabilities are, on average, disclosed only one week after the patch release~\cite{10.1145/3133956.3134072}---leaving, again, users vulnerable and unaware. Therefore, making the information available earlier in the process is of utmost importance.

%%%% not sure how this part helps
% Yet, previous work succeeded in locating software vulnerabilities and their respective patches using manual validation~\cite{10.1109/MSR.2019.00064} or regular expressions~\cite{reis2017secbench, SSPatcher2022} over commit messages (i.e., some data in commit messages can in fact be leveraged for this task).

\textbf{Motivation:} 
Security commit messages (i.e., the commit messages attached to the code changes used to patch software vulnerabilities) can be cryptic~\cite{9678720}, hindering triage tasks. 
Table~\ref{tab:messages} shows examples of security commit messages (col. Message) used to document patches to known software vulnerabilities (col. VulnID) and the 
patch commit key (col. SHA). Message 1 (``.'') is an example of a cryptic message as it does not provide any information. From messages 2 and 3, it
is possible to infer that a defect in code is being fixed, but nothing 
more than that. Message 4 is an internet meme called ``Rickrolling''\footnote{``Rickrolling'' meme details at \url{https://knowyourmeme.com/memes/rickroll}}, usually used to prank people with the "Never Gonna Give You Up" song. Although funny, the message does not provide, again, any relevant information regarding the vulnerability and respective patch. Most of the cases presented would not be detected by triage systems leaving users unaware of the new patches and exposed to vulnerabilities. Message 5  
provides a brief description of the patch and some relevant information about it (e.g., it fixes a ``security issue'' related to ``passwords''). The respective code changes fix a potential buffer overrun vulnerability resulting from reading user-provided passwords and confirmations via command-line prompts. The patch fixes an ``Improper Authentication'' weakness (CWE-287) with a severity (CVSS) score of $8.4$ in $10$---which 
is not explicit in the message. This extra information could have helped
automated or even manual patch triage systems to prioritize this patch since it has high severity. 



\begin{table}[!t]
    \footnotesize
    \centering
    \begin{tabular}{ | p{0.01cm} | p{4.8cm} | p{1.5cm} | p{0.9cm} |} 
    \hline
     & \textbf{Commit Message} & \textbf{VulnID} & \textbf{SHA}\\\hline
        1 & \texttt{.} & CVE-2019-13568 & \href{https://github.com/GreycLab/CImg/commit/ac8003393569aba51048c9d67e1491559877b1d1}{ac80033} \\\hline
        2 & \texttt{Minor patch.} & OSV-2020-2108 & \href{https://github.com/simdjson/simdjson/commit/a8bf10ea5a0ea2553f07ac46744666c94d0085fc}{a8bf10e} \\\hline
        3 & \texttt{Code refactoring.} & GHSA-4fc4-4p5g-6w89 & \href{https://github.com/ckeditor/ckeditor4/commit/d158413449692d920a778503502dcb22881bc949}{d158413} \\\hline
        4 & \texttt{<scratchsig><script>location=
        'https:\/\/www.youtube.com\/watch?
        v=dQw4w9WgXcQ';<\/script><\/scratchsig>} & CVE-2020-15179 & \href{https://github.com//InternationalScratchWiki//wiki-scratchsig//commit//4160a39a20eebeb63a59eb7597a91b961eca6388}{4160a39} \\\hline
        5 & \texttt{Fixed security issues with passwords entered via a prompt} & CVE-2022-35928 & \href{https://github.com/paulej/AESCrypt/commit/68761851b595e96c68c3f46bfc21167e72c6a22c}{6876185} \\
    \hline
    \end{tabular}
    \caption{Examples of commit messages used to patch known software vulnerabilities.}
    \label{tab:messages}
\end{table}


\textbf{Study.} While some argue that security commit messages and patch release notes should be minimalist, others argue that the details are crucial to ensure triage systems effectiveness~\cite{SSPatcher2022,Zhang2021AnIO}, create trust amongst users~\cite{Householder2020}, and enable fast patch management~\cite{Zhang2021AnIO,Householder2020}. Therefore, we investigated the current status of security commit messages and answered the following questions: (1) what information is included in commit messages of public security patches; (2) are security engineers following best practices to produce security commit messages; and, 
(3) is the security community open to a new
standard for security commit messages.
In our work, we empirically analyzed $11036$ security commit messages by extracting key information (security words, vulnerability ID, weakness ID, severity, and more) with a customized named entity recognition (NER) tool.

\textbf{Results.} We found that $61.2\%$ of security commit messages include security-related words but lack key information such as the vulnerability ID, weakness ID, and severity. We were unable to extract any information from $8\%$ of the messages because (1) they were poorly documented, (2) their vocabulary was non-security related, or (3) they had misspelled words. Overall, security engineers poorly follow best practices to write general commit messages indicating that a set of new best practices for this task are needed. 

\textbf{Solution.} To address the challenges identified from our empirical analysis, we developed a structured and comprehensive convention called \texttt{SECOM} for writing security commit messages. We noticed 
security engineers sometimes follow standards to write general commit messages in security commit messages (e.g., Conventional Commits~\cite{convcom}). Therefore, we designed the standard
on top of best practices for writing generic commit messages to facilitate adoption. \texttt{SECOM} was designed according to (1) the results of the empirical analysis of historical commit messages and (2) validation of SECOM with the open-source security community. The standard was created with the aim of making the task of automated reasoning tools easier for patch detection (by adding a clear indicator for vulnerability fixes) and prioritization (by considering severity score and weakness type). 

\begin{figure}[!t]
    \centering
\includegraphics[width=\linewidth]{Figures/cycle.png}
    \caption{Problem and Solution illustration.}\label{fig:cycle}
\end{figure}

\textbf{Impact.} Figure~\ref{fig:cycle} shows where SECOM can help improve patch management processes: well-documented commit messages can be detected by triage systems and shorten the time between the patch is released and deployed. 
From a research perspective, detecting and assessing vulnerabilities continues to be a main challenge in vulnerability prediction due to the scarcity
and poor quality of curated data~\cite{9448435}. 
\texttt{SECOM} can be an important tool in softening this issue since it has the potential to improve the quality of data for future dataset creation. More than 2k security commit messages have been produced with the \texttt{SECOM} convention\footnote{\url{https://github.com/JLLeitschuh/security-research/issues/8}} so far.
We are aware of the challenges that come from too much transparency, and we want to guide the community to make proper and careful use of this new standard. Therefore, in this paper, we also provide guidelines on using \texttt{SECOM} carefully.
% Efforts to automate compliance validation and recommendations are also mentioned in this paper. 

\textbf{Contribution.} In summary, our contributions are the following: 
(1) an empirical analysis of security commit messages and best practices application; (2) a standard for security commit messages, called SECOM; and, (3) guidelines on how to write better security commit messages and apply the convention carefully.



\section{Background}~\label{sec:background}

This section provides background information on the Named Entity Recognition (NER) theory, the approach used to extract key information from security commit messages.

\subsection{Named Entity Recognition}~\label{sec:ner}

\begin{figure}[t!]
\hspace*{-0.25cm}\centering
    \includegraphics[width=\linewidth]{Figures/application.png}
    \caption{Named Entity Recognition (NER) application example for a security commit message.}\label{fig:application}
\end{figure}

Named Entity Recognition (NER) is a form of
Natural Language Processing (NLP)---also referred to as entity chunking, extraction, or identification. It is the task of identifying and extracting key information, called \emph{entities}, from unstructured data (in this case, text)~\cite{9039685, mikheev-etal-1999-named, lample-etal-2016-neural}.

An \emph{entity} can be any word or bag of words that refer to the same \emph{entity category}. For instance, different names of companies ``Netflix'', ``Google'' or ``Apple'' are entities that belong to the \emph{Company} category. 
NER requires the design of specific entity categories and the respective entity values, which relies on good domain knowledge. 
NER has been applied to different domains in the past, such as biomedical sciences~\cite{hakala-pyysalo-2019-biomedical}, pharmacology~\cite{gonzalez-agirre-etal-2019-pharmaconer}, and even security. 

\subsection{Named Entity Recognition in Security}~\label{sec:ner}

Previous work used NER to extract product names and versions from vulnerability reports~\cite{10.5555/3361338.3361399}. In our empirical analysis, we designed a group of category entities that are usually found in security commit messages and used a customized NER tool for security to extract key information (or entities) from commit messages used to patch software vulnerabilities.
Figure~\ref{fig:application}
shows an example of how we applied NER to security commit messages. Our tool extracted 4 different entities (``fixed'', ``security'', ``issue'', ``passwords'') for 3 different category entities (``ACTION'', ``SECWORD'', ``FLAW''). Only relevant words are extracted and tagged with their respective category entities, which is crucial for a  better understanding of the meaningful information provided in security commit messages.


\section{Research Questions}\label{sec:research_questions}

We use a mixed-method approach to answer three research questions. The first two research questions, RQ1 and RQ2, are answered by a cross-sectional empirical study of existing security commit messages. The third research question, RQ3, is answered through a survey study. 

\subsubsection*{\textbf{RQ1. What information is included in
the commit messages of public security patches?}}

Previous work has shown that SVM techniques can not rely purely on commit messages due to poor data quality~\cite{DBLP:journals/corr/abs-1806-05893}.
A study on vulnerabilities patched ``silently'' showed that only $38\%$ of commit messages used to fix vulnerabilities included security-related words~\cite{9678720}. 
But some previous approaches have managed to find some relevant natural language information in commit messages and perform security patch detection~\cite{reis2017secbench,10.1145/3106237.3117771,SSPatcher2022,DBLP:journals/corr/abs-1807-02458,10.1145/3593434.3593481}. The question 
that remains is what information is being mentioned in the commit messages
of security patches.

% \textbf{RQ2. Are there any guidelines available to
% produce security commit messages?} One way to produce quality 
% commit messages is by following best practices or guidelines~\cite{Tian_2022, convcom,linus,atomic}. In this part
% of the study, we searched for potential 
% standards to write commit messages of software security patches.

\textbf{RQ2. Do security engineers follow best practices to write security commit
messages?} 
One way to produce quality 
commit messages is by following best practices or guidelines~\cite{Tian_2022, convcom,linus,atomic}. However, researchers found that $44\%$ of commit messages need improvement~\cite{Tian_2022}. 
Our research revealed a lack of standards for writing security commit messages. Instead,
we found standards and guidelines for generic commit messages~\cite{convcom, linus, goodcommit}.
%In this part of our empirical analysis, 
Therefore, we explored if available guidelines or standards 
were being used and how they could be leveraged to create more structured and complete commit messages for security patches.

\emph{\textbf{RQ3. How open is the security community to a new standard for security commit messages?}}. Standards are usually seen as a burden or with resistance. However, in order to fix challenges, we need to create them. Therefore, we validated our convention with the open-source security community (Open Source
Security Foundation).


\section{Study of Existing Security Commit Messages}\label{sec:study_design}

This section describes an exploratory, cross-sectional empirical study of security commit messages collected from security patches for known vulnerabilities for answering RQ1 and RQ2. 


\subsection{Dataset Collection and Preprocessing} 

This section presents the steps taken to create a dataset of security commit messages.
All the tools used to collect and preprocess the data are available in our replication package. Our dataset considers data released until the 12th of August, 2022.
We collected security commit messages for our exploratory analysis by following the steps described below.

\subsubsection{Vulnerability Metadata Collection from Public Vulnerability Databases}
%
Public vulnerability databases, such as 
the National Vulnerability Database 
(NVD)~\cite{nvd}, and
the Open-Source Vulnerability (OSV) 
database~\cite{osv}, integrate documentation (or reports) for thousands of known vulnerabilities. Our dump of the OSV database includes a total of $30091$ vulnerability reports for open-source vulnerabilities from different ecosystems:
$28.6\%$ of the vulnerabilities were reported by GitHub Advisories, $25.7\%$ by  Linux, $11.1\%$ by PyPI,
$8.5\%$ by NPM, $7.8\%$ by OSS-Fuzz, and, the remaining $18.3\%$ by the rest of the
sources (e.g., Maven, RubyGems, Go,
and more). OSV only includes reports of vulnerabilities published after $2005$ (inclusive) 
and vulnerabilities reported by a restricted group of ecosystems. NVD includes reports of known vulnerabilities published since $1999$ and has no restrictions regarding the 
ecosystem (as far as we know). Therefore, we also considered the 
NVD database in our study. For NVD, we collected a total of $181614$ 
vulnerability reports. In total, we collected $211705$ vulnerability reports from both databases.


\begin{table*}[t!]
\footnotesize
    \centering
        \caption{Entity category names, rationale, and entity examples.} 
    \begin{tabular}{ | p{0.5cm} | p{1.4cm} | p{8.5cm} | p{5.5cm} |  p{0.5cm} | }
    \hline
         \textbf{Type} & \textbf{Category} & \textbf{Rationale} & \textbf{Entity Examples} & \textbf{Rules} \\\hline
        
        \multirow{5}{*}{SEC} & SECWORD & Security-relevant words are usually used to describe the vulnerability and respective fix (we used a large set of security-relevant words collected in previous work~\cite{10.1145/3133956.3134072,10.1145/3475716.3475781}). & ldap injection, crlf injection, improper validation, command injection, cross-site scripting, sanitize, bypass & 1719 \\\cline{2-5}
        
        & VULNID & Vulnerability IDs are used to identify vulnerabilities for different ecosystems in commit messages: CVE, GHSA, OSV, PyPI, etc. We crafted rules for the different IDs patterns. & GHSA-269q-hmxg-m83q, CVE-2016-2512, CVE-2015-8309, GHSA-9x4c-63pf-525f, OSV-2016-1 & 9
        \\\cline{2-5}
        
        & CWEID & Vulnerabilities usually belong to a weakness type. One common taxonomy used to classify security weaknesses is the Common Weakness Enumeration (CWE) one. Therefore, we crafted rules to detect CWE IDs. & CWE-119, CWE-20, CWE-79, CWE-189 & 2 \\\cline{2-5}
        
        & SEVERITY & Vulnerabilities usually have a severity assigned. & low, medium, high, critical & 4\\\cline{2-5}
        
        & DETECTION & Vulnerabilities are detected manually or using specific tools. & Manual, CodeQL, Coverity, OSS-Fuzz, libfuzzer & 8\\\hline
        
        \multirow{7}{*}{COM} & SHA & Commit hashes that reference older versions where the vulnerability was introduced (OSV Schema~\cite{osv-schema}). &  f8d773084564, 228a782c2dd0 & 2
        \\\cline{2-5}

        & ACTION  & A commit usually implies an action, in the case of security, 
        fixing a vulnerability (corrective maintenance). & fix, patch, change, add, remove, found, protect, update, optimize, mitigate & 18\\\cline{2-5}
        
        & FLAW & Fixing a security vulnerability usually implies fixing a flaw. & defect, weakness, flaw, fault, bug, issue & 10 \\\cline{2-5}
        
        & ISSUE & The GitHub issue/pull request number is sometimes referenced in the message and can provide more information on the vulnerability. & \#2, \#13245 & 1 \\\cline{2-5}
        
        & EMAIL & Contact e-mails of reviewers and authors usually appear after tags such as `Reported-by` and are important to know who to contact. & \url{johndoe123@gmail.com, catlover@yahooo.com, adventuretime@hotmail.com, supercool@outlook.com}$^1$  & 1\\\cline{2-5}
        
        & URL & Links to reports, blog posts, and bug-trackers references provide more information about the vulnerability. & \url{https://www.htbridge.ch/advisory/multiple_vulnerabilities_in_mantisbt.html} & 1\\\cline{2-5}
        
        & VERSION & Software versions are commonly referenced in commit messages. & 3.1.0, v3.2, v2.6.28, 1.6.3, 2.1.395 & 4\\\hline
        \multicolumn{5}{|l|}{\textbf{Type SEC}: Security specific entity categories; \textbf{Type COM}: Commit specific entity categories.} \\
        \multicolumn{5}{|l|}{$^1$\textbf{Artificial} e-mails generated automatically with ChatGPT for compliance with General Data Protection Regulation (GDPR).} \\\hline
    \end{tabular}
    \label{tab:entities-desc}
\end{table*}

\subsubsection{Collection and Preprocessing of References to Security Patches}

Each vulnerability report for both data sources includes a section referencing the fix (or patch), when 
available. An example of such a section can be found in~\cite{cve-example}. 

\textbf{Collection:} To get the commits involved in patching the security vulnerability, we filtered out all the vulnerability reports without references to commit links. We discovered that only $10010$ out of $30091$ OSV reports and $9953$ out of $181614$ NVD reports have references to commits 
(i.e., only $9\%$ of the vulnerability reports in those databases have fixes available). In addition, we observed that commits 
are usually available through GitHub, Bitbucket, SVN, and other services.
In this study, we only focus on vulnerability reports that include commits for fixes (i.e, security patches) available on GitHub, which accounts for over $80\%$ of the commits extracted from vulnerability 
reports. In total, we found references to GitHub fixes in $8670$ NVD reports and $9576$ OSV reports.

\textbf{Preprocessing:}
In order to collect the commit message of these commits, we used the GitHub API, which requires knowing the \textit{owner} of the repository that integrates the commit; the \textit{name of the repository}; and, the \textit{version} (or, \texttt{SHA} key) that included the vulnerability. 
However, sometimes due to the lack of precise information, we could not determine the data required to get the commit message (e.g., when the commit link had master instead of a specific \texttt{SHA} key). Thus, we could not ensure that the current version on master was the version where the vulnerability had been detected. Therefore, we removed all the vulnerability reports exhibiting  this issue,  which resulted in a total of $8405$ security patches for NVD ($3\%$ of data points) and $9466$ security patches for OSV ($1\%$ of data points). 

\textbf{Merging and Cleaning:}
Both data sources were merged after normalization into a dataset of $17871$ security patches while keeping the vulnerability reports metadata. Many vulnerabilities 
are reported in both NVD and OSV. 
Therefore, we found duplicates between both sources using different heuristics: 1) duplicated entries for security patches but with missing values for vulnerability score in one of the sources ($18\%$ of data points); 2) OSV reports contain a field called ``aliases'' which is a list of IDs of the same vulnerability in other databases. Therefore, we removed all the NVD entries ($19\%$ of data points) whose IDs where already in the aliases of OSV reports---OSV data was prioritized since previous research has shown that NVD has documentation problems and OSV is making an effort to fix those problems;
3) vulnerabilities fixed with the same 
patch, usually vulnerabilities that affect 
different codebases and therefore result in different vulnerability reports were also removed ($13\%$ of data points).
After removing the different types of duplicates, we end up with a dataset of $10254$ security patches.


\subsubsection{Collection and Preprocessing of Security Commit Messages}

Vulnerabilities can be fixed with one commit (single-commit patch) or multiple commits (multi-commit patch)---$88.6\%$ ($9083$) of the patches are single-commit patches while the other $11.4\%$ ($1170$) are multi-commit patches. From $10254$ security patches, we extracted a total of $11809$ security commits. GitHub metadata (including the commit message) was collected using the GitHub API. The commit messages were preprocessed in different ways: \textbf{(1)} A total of $334$  
commits (the equivalent to $160$ vulnerability reports) were \emph{no longer available} at the metadata collection time. Therefore, they were removed from the dataset. \textbf{(2)} We found \emph{duplicated commit messages} resulting from vulnerability reports with references to the vulnerability fix but deployed in different branches. One example is the GHSA-273r-mgr4-v34f\footnote{https://github.com/advisories/GHSA-273r-mgr4-v34f}, which references a commit per branch where the vulnerability was fixed. In these cases, since the commit messages are the same, we only kept one of the commits. Therefore, an extra $270$ commits were removed from the dataset---which left us with $11205$ security commit messages. \textbf{(3)} As in previous work~\cite{Tian_2022}, we searched for the same \emph{non-human generated message patterns} except when the original commit message was somehow attached to the commit message under analysis. For instance, in cases with the pattern ``<original commit message> (cherry picked from commit <commit>)'', the original commit message is attached at the beginning, and, in the ``merge pull request ... <original commit message>'' pattern, the original commit message is attached at the end. Therefore, we only remove non-human written messages that do not include any text generated by humans. One example is the \texttt{GHSA-3m93-m4q6-mc6v}\footnote{https://github.com/advisories/GHSA-3m93-m4q6-mc6v} advisory, which only references the cherry-picked commit. In addition to the patterns mentioned in~\cite{Tian_2022}, we also removed commit messages with pull request merges from dependabot and merge pull requests without any human text such as ``merge pull request from ghsa-g4hm-6vfr-q3wg''.
In summary, we found and removed a total of $126$ automated commit messages and kept $11079$ security commit messages. \textbf{(4)} We noticed some of the 
commit messages were not written in English. We ran \texttt{langdetect}\footnote{\texttt{langdetect}is an algorithm to infer the natural text language. It supports $55$ different text languages. Available at \url{https://github.com/Mimino666/langdetect}.} to infer the message's language. The model detected $1311$ ($11.8\%$) security commit messages as non-English. The tool can perform inaccurate predictions when evaluating too short or too ambiguous text. Therefore, we manually inspected the non-English messages to make sure we would not remove English and valid messages. 
After manual validation, we removed an extra $43$ non-English messages such as ``\begin{CJK*}{UTF8}{gbsn}导入mysql db时报错\end{CJK*}'' or ``\begin{CJK*}{UTF8}{gbsn}用户头像上传格式限制\end{CJK*}''.

\textbf{Results.} We successfully collected a total of $11036$ security commit messages (corresponding to $9943$ security patches). This dataset includes security commit messages used 
to document patches for $278$ different security weaknesses.


\subsection{Data Extraction (R1)}

This section explains the methodology used to extract key information from commit messages and answer our first research question. 

\subsubsection{Defining Domain-Specific Entities and Categories} 
% NER is an effective NLP technique to identify and tag entities based on specific rules or parameters~\cite{mikheev-etal-1999-named}. While 
% Text Classification looks at the characteristics of the text as a whole to draw conclusions (e.g., sentiment analysis), NER can provide more understanding of the text structure through the extraction of 
% key information (also known as entities). Previous work in the field has focused mainly on Text Classification~\cite{}. Both are important, since 

% To answer the question, ``\textbf{RQ1: Are security commit messages 
% informative?}'', w
We used NER (see Section~\ref{sec:ner}) to 
extract key information from security commit messages.
%
Firstly, we designed different entity categories: (1) security-specific or Type SEC, i.e., groups of words or bags-of-words that are common in security commit messages and security vulnerability reports (e.g., SECWORD, VULNID, CWEID, and more); and (2) commit-specific or Type COM (e.g., SHA, ACTION, FLAW, ISSUE, and more), i.e., groups of words or bags-of-words that are common in general commit messages. 
Table~\ref{tab:entities-desc} describes the different entity 
categories, the reason why each of them was considered (\textit{rationale}), entity examples, and the number of rules used to extract data each entity category---which can be fully inspected in our replication package or briefly in Table~\ref{tab:rules}. 

\subsubsection{Named Entity Extraction Pipeline} To extract the 
entities for each category, we used a Python library called Spacy\footnote{https://spacy.io/}---which provides end-to-end 
pipelines for several natural language processing tasks (e.g., NER). We built our own customized NER pipeline for security (see Figure~\ref{fig:parsing}). \textit{\textbf{Tokenization.}} Our pipeline 
takes as input a security commit message that is tokenized (or split into meaningful segments, called \emph{tokens}) using Spacy's tokenizer\footnote{Spacy's tokenizer documentation: \url{https://spacy.io/api/tokenizer}. More 
details on how it works here: \url{https://spacy.io/usage/linguistic-features\#tokenization}.}.
\textit{\textbf{Part-Of-Speech Tagging.}} Then, the tokens are tagged using the Part-Of-Speech (POS) 
tagger\footnote{Spacy's Part-Of-Speech (POS) tagger documentation: 
\url{https://spacy.io/api/tagger}.}, a pre-trained pipeline component to predict part-of-speech 
tags\footnote{Universal POS tags available at \url{https://universaldependencies.org/u/pos/}} 
such as verb, noun, adjective, adverb, and so on.
Pre-trained pipeline components to predict POS
tags are language-dependent. Since we focus on English text, 
we used a pre-trained model for English 
 (\texttt{en\_core\_web\_lg}\footnote{\url{https://spacy.io/models/en\#en\_core\_web\_lg}}) to extract the POS tags.
After collecting the tokens and, respective, POS tags, the 
pipeline applies a customized set of rules for security commit messages. Spacy's entity ruler\footnote{Spacy's Entity Ruler documentation: \url{https://spacy.io/usage/rule-based-matching\#entityruler}.} enables the customization 
of entity recognition in text. 

\begin{figure}[t!]
\hspace*{-0.25cm}\centering
    \includegraphics[scale=0.36]{Figures/parsing.png}
    \caption{Extraction Pipeline}\label{fig:parsing}
\end{figure}


\subsubsection{Rules} Fixing security vulnerabilities 
usually involves an ``ACTION'' (one of our category entities, Table~\ref{tab:entities}). Actions can be represented in natural language with verbs such as ``fix'', ``update'', ``patch'', ``mitigate'' and more. Therefore, we created several rules that search for different verbal forms of these words. 
Table~\ref{tab:rules} shows two examples of rules used by our entity ruler. The first rule extracts all the different verbal forms of the word ``fix'' (e.g., ``fix'', ``fixing'', ``fixed'', ``fixes''), i.e., it extracts all variations of the token ``fix'' when its POS tag is a \texttt{VERB}. 

\begin{table}[t!]
\footnotesize
    \centering
        \caption{NER Rule examples. All rules are 
        available in our replication package 
        (\texttt{entity\_ruler/patterns.jsonl}).} 
    \begin{tabular}{ | p{0.25cm} | p{1.25cm} | p{6cm} | }
    \hline
        \textbf{ID} & \textbf{Label} & \textbf{Rule}\\\hline
        R1 & ACTION & \verb|{"label":"ACTION","pattern":[{"LOWER": |\\& & \verb| {"REGEX":"fix.*"},"POS":"VERB"}]}|  \\\hline
        R2 &  VULNID & \verb|{"label":"VULNID","pattern":[{"LOWER":|\\& & \verb|"osv"},{"IS_PUNCT":true,"OP":"?"},|\\& & \verb|{"LOWER":{"REGEX":"\\d{4}"}},|\\& & \verb|{"IS_PUNCT":true,"OP":"?"},|\\& & \verb|{"LIKE_NUM":true}],"id":"OSV"}| 
        \\\hline
    \end{tabular}
    \label{tab:rules}
\end{table}

The second example
illustrates the extraction of different vulnerability IDs. With the 
growth of the open-source security community, different 
ecosystems (e.g., PyPI, NPM, Ruby Gems, and more) are 
starting to report vulnerabilities with their own IDs that 
follow different structures than the usual CVE ID, \verb|CVE-\d-\d{4,7}|. Therefore, we implemented rules to extract the 
different vulnerability IDs (VULNID). One example is R2, 
a rule to extract vulnerability IDs of type (or rule id) OSV for vulnerabilities 
detected with Google's OSS-Fuzzer (e.g., OSV-2023-
27\footnote{https://osv.dev/vulnerability/OSV-2023-27}). 

In  addition, we created a total of $1719$
rules---from words or 
bags of words collected in previous 
work~\cite{10.1145/3133956.3134072,10.1145/3475716.3475781}.  These rules extract security-related words (\texttt{SECWORD}). Many rules were improved after manual 
inspection of commit messages, where (1) we detected erroneous 
extraction when looking at the entities (e.g., due to broken 
tokenization of GitHub Advisory IDs, we had to create several 
different rules based on different tokenization results); or, 
(2) the tool did not extract any entity. This process led us to 
augment the list of category entities and find a set of 
anti-patterns for security commit messages---which is described in detail on Table~\ref{tab:fields}. 


\subsection{Best Practices Analysis (RQ2)}

One way to produce quality 
commit messages is by following best practices or guidelines~\cite{Tian_2022, convcom,linus,atomic}. 
In this part of the study, we explored if some of the available guidelines or standards 
were already under usage and how they could be leveraged to create more structured and complete commit messages for security patches.
We found six main suggestions to produce good commit messages from four sources of guidelines~\cite{convcom,goodcommit,linus,Tian_2022}: (1) conventional commits suggest adding a type as a prefix to the subject/header such as \texttt{fix:} or  \texttt{feat:}~\cite{convcom}; (2) the header should explain the commit in one line and meaningful~\cite{convcom,goodcommit} in the capitalized form, no period in the end and the imperative form~\cite{goodcommit}; (3) the body should explain the problem (what), its impact (why) and the fix (how)~\cite{linus,Tian_2022}; (4) the message should include references to bug-trackers, issues or pull requests (when GitHub is used to manage defects)~\cite{goodcommit}; (5) contacts of the reviewers and reporters~\cite{linus}; and finally, (6) keep commits atomic, i.e., one task per commit~\cite{goodcommit}.

We translated the suggestions mentioned before into seven different compliance checkers, which we used to assess if security engineers are following best practices. Table~\ref{tab:practices} shows the different compliance checkers and their sources. %



\begin{table}[t!]
    \footnotesize
    \centering
        \caption{Best Practices to Write Generic Commit Messages} 

    \begin{tabular}{| p{0.25cm} | p{6cm} | p{1.25cm} | }
    \hline
        \textbf{ID} & \textbf{Best Practice} &  \textbf{Standard}
        \\\hline
        C1 &  The header should be prefixed with a type.  & ~\cite{convcom} \\\hline
        C2 & The message should have a one-line header/subject.  & ~\cite{convcom, linus, goodcommit}\\\hline
        
C3 & The message should have a body.  & ~\cite{linus, goodcommit}\\\hline

C4 & The message should mention the contact of the author (signed-off-by and authored-by).  & ~\cite{linus, goodcommit}\\\hline

C5 & The message should mention the contact of the reviewer (reviewed-by).  & ~\cite{linus, goodcommit}\\\hline

C6 & The message should mention references to issues or pull requests.  & ~\cite{goodcommit}\\\hline

C7 & The message should include references to bug trackers.  & ~\cite{goodcommit}\\\hline

    \end{tabular}
    \label{tab:practices}
\end{table}

\section{SECOM: A Convention for Security Commit Messages and its Validation}~\label{sec:solution}

To address the problems mentioned in the previous section, we propose SECOM, a convention for writing security commit messages; and, validate
its usage with the security 
community. The new convention helps us answer the research question RQ3.  


\subsection{Design of the Convention}

In our empirical study, we observed that although security commit messages do not follow best practices in general, there is a small percentage of people using them---i.e., best practices are being used by some security engineers even if only by a small percentage (Section~\ref{sec:findings}). To facilitate adoption, we created \texttt{SECOM} on top of a well-known group of 
guidelines
for writing good commit messages~\cite{convcom, atomic, linus, goodcommit}.  
Table~\ref{tab:fields} lists and describes the different fields and the reason why each field was considered (\textit{rationale}). For instance, the ``type'' field 
was considered because, according to the Conventional Commits Specification, the header/subject should start with a type ($4.10\%$ of security commit messages include a type at the beginning of the header). In addition, the Google OSV team suggested that type should be indeed considered and proposed a new word for security patches, ``vuln-fix''.

The structure and set of fields 
included in the convention were inferred 
from (1) our empirical analysis  of security
commit messages collected from security patches available in vulnerability
databases such as NVD and OSV; 
(2) feedback collected alongside
the Open Source Security Foundation community; and, (3) previous work on best practices for commit messages~\cite{convcom, atomic, linus, goodcommit,Tian_2022}.

The convention (Listing~\ref{lst:SECOM}) consists of five main 
sections: \textbf{header}, prefixed with the type \texttt{vuln-fix}, 
a simple description of the vulnerability and its identifier 
(when available); \textbf{body}, describes the vulnerability 
(what), its impact (why) and the patch to fix the vulnerability 
(how); \textbf{metadata}, such as type of weakness (CWE-ID), 
severity, CVSS, detection methods, report link, and version 
of the software where the vulnerability was introduced; 
\textbf{contacts}, the names and e-mail contacts of 
the \textbf{reporters} and \textbf{reviewers}; and, finally,
\textbf{references} to bug trackers. The different
sections should be separated with a new line.

\begin{lstlisting}[caption={SECOM Convention},label={lst:SECOM},frame=tlrb]
<type>: <header/subject> (<Vuln-ID>)

<body>
# (what) describe the vulnerability
# (why) describe its impact
# (how) describe the patch/fix

[For Each Weakness in Weaknesses:]
Weakness: <Weakness Name or CWE-ID>
Severity: <Low, Medium, High, Critical>
CVSS: <Severity Numerical Repr. (0-10)>
Detection: <Detection Method>
Report: <Report Link>
Introduced in: <Commit Hash>
[End]

Reported-by: <Name> (<Contact>)
Reviewed-by: <Name> (<Contact>)
Co-authored-by: <Name> (<Contact>)
Signed-off-by: <Name> (<Contact>)

Bug-tracker: <Bug-tracker Link>
OR
Resolves: <Issue/PR No.>
See also: <Issue/PR No.>
\end{lstlisting}

The convention considers security patches should be atomic~\cite{atomic}, i.e., two weaknesses can be patched by the same fix, but if it requires more than one fix, then it should be a different commit. 


\begin{table*}
    \footnotesize
    \centering
    \begin{tabular}{ | p{1.75cm} | p{4cm} | p{11cm} | } 
    \hline
        \textbf{Field} & \textbf{Description} & \textbf{Rationale}\\\hline
        type & Usage of \texttt{vuln-fix} at the beginning of the header/subject to specify the fix is related to a vulnerability. & A \textbf{type} should be assigned to each commit~\cite{convcom}---which will make the identification of vulnerability fixes easier. The \texttt{vuln-fix} value was proposed by the Google OSV team during the feedback collection \textbf{(F)} phase. In addition, 4.10\% of commits follow the
conventional commits convention “<type>(scope):”. \\\hline
        Header/Subject & It should be approximately 50 chars (max 72 chars), capitalized with no period in the end and in the imperative form. & According to the common best practices for commit messages, it is important to summarize the purpose of the commit in one line~\cite{linus, goodcommit}. In our best practices analysis, we observed that $100\%$ of commit messages had a header, but only $38.85\%$ had security-related words and represented an action. \\\hline
        Vuln-ID & When available, e.g., CVE, OSV, GHSA, and other formats. & Adding the vulnerability ID to the header/subject can help to localize the commit responsible for patching the vulnerability faster using features like reflog or shortlog. Only 12.1\% of commit messages included mentions of the vulnerability ID, but 4 out of the 7 participants in \textbf{(F)} phase found including the vulnerability ID in the message important.\\\hline
        Body &  Describe the vulnerability (what), its impact (why), and the patch
    to fix the vulnerability (how) in approximately 75 words (25 words per point). & The body is the most important part of the commit message since it provides space to add details on the problem, impact, and solution~\cite{Tian_2022}. In our empirical analysis, we observed that  59.91\%
    commit messages have a body. However,  only 4031 out of those 6875 cases included security-related words or had meaningful information. \\\hline
        Weakness &  Common Weakness Enumeration ID or name. & The weakness ID provides information on which type of vulnerability can exist in the software. Software patch management teams may proceed differently according to the type of weakness. However, only 0.2\% of messages included this type of information. \\\hline
        Severity &  Severity of the issue (Low, Medium, High, Critical). & Severity can motivate software users to perform patch management faster (in case of critical vulnerabilities)~\cite{Householder2020}. Again, only 1.1\% of commit messages mentioned severity levels. \\\hline
        CVSS &  Numerical (0-10) representation of the severity of a security 
    vulnerability (Common Vulnerability Scoring System). & CVSS allows users to make better sense of the vulnerability severity and can motivate software users to perform patch management faster~\cite{Householder2020}. This field was proposed by a security engineer at OpenSSF that mentioned that sometimes is possible to calculate the score by following the CVSS questionnaire. \\\hline
        Detection &  Detection method (Tool, Manual, etc). & It can be interesting to help future researchers with replication. $4$ out of the $7$ participants in the \textbf{(F)} phase sees value in adding this field (Table~\ref{tab:survey}, RQ2). \\\hline
        Report & Link for vulnerability report, which can back up the lack of information provided in commit messages. & It usually provides more information on the vulnerability exploit or proof-of-concept. We observed that 3 out of the 7 participants would like to see links to reports, \textbf{(F)} phase (RQ1). \\\hline
        Introduced in & Commit hash from the commit that introduced the 
    vulnerability. & Suggested by a survey participant of the (\textbf{F}) phase and used in the OSV Schema~\cite{osv-schema}. In addition, we found SHA keys in 1467 commit messages.\\\hline
        Signed-off by & Name and contact of the person that reported the issue.  & To provide credit to the person that found the problem and ask for more details when necessary. However, only 8.4\% of commit messages were signed off by the respective authors.\\\hline
        Reviewed-by & Name and contact of the person that reviewed and closed the issue. & Reviewers are usually the internal developers or senior developers that review and approve the issues. Only 3.33\% of messages have the reviewers' contact.\\\hline
         Bug-tracker & Link to the issue in an external bug-tracker or \texttt{Resolves... See also:} when GitHub is used to manage issues. & Important to document and discuss the problem, its impact and know which people were involved. In our empirical analysis, we extracted URLs from a total of $929$ commits. \\  
    \hline
    \end{tabular}
    \caption{Fields description and rationale.}
    \label{tab:fields}
\end{table*}

% \textbf{Threats to Validity.} Our sample of participants is small, which may not reflect the entire population. In addition, we only considered participants from the open-source community (i.e., private software developers may not be represented). 

% \textbf{Initial Impact.} SECOM was mentioned as one of 
% the best practices for bulk generation of pull requests 
% to scale vulnerability patching in conferences such as BackHat and Defcon~\cite{blackhat}.


\subsection{Compliance Checklist}

Table~\ref{tab:checklist} provides a checklist (or set of rules) to produce better security commit messages.
For each section of the convention, practitioners will find the fields that should be added to the 
security commit message and questions they should ask when filling them out. We find all the fields important; however, for the sake of prioritization when time is short, we selected some of them as mandatory---which are the ones we think to be most important to detect (type, Vuln-ID), prioritize (Weakness, Severity, CVSS) and understand the message (header, what, why, how). However, practitioners should always try to make security commit messages as detailed as possible. The compliance validation with \texttt{SECOM}'s structure (Listing~\ref{lst:SECOM}) and rules (Table~\ref{tab:checklist}) is currently automated with a tool. In the future, we plan to explore how to generate suggestions to produce better security commit messages or even entire ones to soften the burden of a new standard in maintenance teams.



\begin{table*}
    \footnotesize

    \begin{tabular}{ | p{1.75cm} | p{1.75cm} | p{11.75cm} | p{0.25cm} | } 
    \hline
        \multirow{4}{*}{\textbf{Header}} & type & Did you set the type of the commit as "vuln-fix" at the beginning of the header? & \textbf{M} \\\cline{2-4}
                                & header/subject & Did you summarize the patch changes? & \textbf{M} \\\cline{2-4}
                                & header/subject & Did you summarize the patch changes within $\sim$50 chars?	 & \textbf{O} \\\cline{2-4}
                                & Vuln-ID & Is there a vulnerability ID available? Did you include it between parentheses at the end of the header? & \textbf{M} \\
    \hline\hline
        \multirow{4}{*}{\textbf{Body}} & what & Did you describe the vulnerability or problem in the first sentence of the body?	 & \textbf{M} \\\cline{2-4}
                                & why & Did you describe the impact of the vulnerability in the second sentence of the body?	 & \textbf{M} \\\cline{2-4}
                                & how & Did you describe how the vulnerability was fixed in the third sentence?		 & \textbf{M} \\\cline{2-4}
                                & * & Did you describe the what, why, and how within $\sim$75 words ($\sim$25 words per section)? & \textbf{O} \\
    \hline\hline
        \multirow{6}{*}{\textbf{Metadata}} & Weakness & Can this vulnerability be classified with a type? If so, add it to the metadata section. & \textbf{M} \\\cline{2-4}
                                & Severity & Can infer severity (Low, Medium, High, Critical) for this vulnerability? If so, add it to the metadata section.	 & \textbf{M} \\\cline{2-4}
                                & CVSS & Can you calculate the numerical representation of the severity through the Common Vulnerability Scoring System calculator (https://www.first.org/cvss/calculator/3.0)?			 & \textbf{M} \\\cline{2-4}
                                & Detection & How did you find this vulnerability? (e.g., Tool, Manual, etc) & \textbf{O} \\\cline{2-4}
                                & Report & Is there a link for the vulnerability report available? If so, include it. & \textbf{O} \\\cline{2-4}
                                & Introduced in	 & Include the commit hash from the commit where the vulnerability was introduced. & \textbf{O} \\
    \hline\hline
        \multirow{2}{*}{\textbf{Contacts}} & Reviewed-by & Include the name and/or contact of the person that reviewed and accepted the patch.  & \textbf{O}\\\cline{2-4}
                                    & Signed-off-by	 & Include the name and/or contact of the person that authored the patch.	  & \textbf{M}\\
    \hline\hline
        \multirow{2}{*}{\textbf{Bug-Tracker}} & External & Include the link to the issues or pull requests in the external bug-tracker. & \textbf{O}\\\cline{2-4}
                                    & GitHub	 & Include the links for the issues and pull-requests related to the patch (\texttt{Resolves.. See also:}).	  & \textbf{O}\\
    \hline
    \end{tabular}
    \caption{SECOM Compliance Checklist. [\textbf{M}-Mandatory; \textbf{O}-Optional; \textbf{*}-All fields in the section.]}
    \label{tab:checklist}
\end{table*}

\subsection{SECOM's Application Example}

\texttt{SECOM} is meant to structure and help write more informative and meaningful
security commit messages. In this section, we show an application example of \texttt{SECOM}
to improve the commit message used to document the patch to CVE-2012-0036\footnote{https://nvd.nist.gov/vuln/detail/CVE-2012-0036}, a potential data injection via a crafted URL.

\begin{lstlisting}[caption={Original commit message to fix CVE-2012-0036},label={lst:before},basicstyle=\scriptsize,frame=tlrb]
URL sanitize: reject URLs containing bad data
Protocols (IMAP, POP3 and SMTP) that use the path 
part of a URL in a decoded manner now use the new 
Curl_urldecode() function to reject URLs with 
embedded control codes (anything that is or 
decodes to a byte value less than 32).

URLs containing such codes could easily otherwise 
be used to do harm and allow users to do 
unintended actions with otherwise innocent tools 
and applications. Like for example using a URL 
like pop3://pop3.example.com/1%0d%0aDELE%201 
when the app wants a URL to get a mail and 
instead this would delete one.

This flaw is considered a security 
vulnerability: CVE-2012-0036

Security advisory at: 
http://curl.haxx.se/docs/adv_20120124.html

Reported by: Dan Fandrich
\end{lstlisting}

\begin{lstlisting}[caption={Commit message to fix CVE-2012-0036 (after SECOM's application)},label={lst:after},basicstyle=\scriptsize,frame=tlrb]
vuln-fix: Sanitize URLs to reject malicious data 
(CVE-2012-0036)

Protocols (IMAP, POP3 and SMTP) that use the path 
part of a URL in a decoded manner now use the new 
Curl_urldecode() function to reject URLs with 
embedded control codes (anything that is or 
decodes to a byte value less than 32).
URLs containing such codes could easily otherwise 
be used to do harm and allow users to do 
unintended actions with otherwise innocent tools 
and applications.
Like for example using a URL like 
pop3://pop3.example.com/1%0d%0aDELE%201 when the 
app wants a URL to get a mail and instead this 
would delete one.

Weakness: CWE-89
Severity: High
Detection: Manual
Report: https://curl.se/docs/CVE-2012-0036.html

Reported-by: Dan Fandrich
Signed-off-by: Daniel Stenberg (daniel@haxx.se)

Resolves: #17940
See also: #17937
\end{lstlisting}

Listing~\ref{lst:before} shows the original commit message used to document the code changes
to patch the CVE-2012-0036. Listing~\ref{lst:after} shows an example of the same commit message, 
but if the message had followed the SECOM standard convention and rules. As seen, the header
includes a type, a short message, and the vulnerability ID, followed by the body where a description
of the problem and fix is provided, and later all the information regarding metadata, contacts, and bug trackers are also provided. In the original message, details such as weakness and severity were not provided, which would not allow triage systems to prioritize this patch properly.

\subsection{Standard Validation (RQ3)}

This section describes how we collected feedback (F) from the security community about the proposed convention (SECOM). We prepared a small survey to collect feedback from the open-source security community. 

The survey did not collect any personal data, and we informed that to our participants at the beginning of the form. \texttt{SECOM} was presented and discussed in detail in two working groups of the Open-Source Security Foundation (OpenSSF). 
OpenSSF is an open community where 
security engineers from all over the industry are involved (e.g., Google, Linux Foundation, Intel, RedHat, and more) and on a mission to 
make open-source security better~\cite{OPENSSF-MISSION}. The convention was validated by experienced security engineers involved in the best practices and vulnerability disclosure projects.
At the end
of our discussion, we asked the present security engineers
to provide their feedback on the convention by answering a few questions (see
Table~\ref{tab:survey}). These questions were designed to validate the different fields
we proposed (Q1 and Q2) and to understand the community's opinion on our new solution (Q3-Q5).
For instance, we wanted to understand which field the community finds more important. One 
thing we observed in our manual analysis of commit messages was the mention of the 
process of detection of the vulnerability---sometimes, the tool used to detect the 
vulnerability was mentioned in the message. Since we did not have any strong 
data analysis to backup this decision, we asked the potential users what they 
thought about it.

\begin{table}[!b]
    \footnotesize
    \centering
    \begin{tabular}{ | p{0.25cm} | p{4.5cm} | p{1.75cm} | } 
    \hline
        \textbf{No.} & \textbf{Question} & \textbf{Answer}\\\hline
       Q1 & ``The following pieces of information commonly appear in commit messages of security-related patches.  Which do you find important to include in a security commit message?''  & Vuln-ID, Severity, CVSS, Weakness Type, Report Link\\\hline
        Q2 & ``Security vulnerabilities are usually detected by different means (e.g., manually, static analysis, dynamic analysis, penetration testing, etc.). Do you see value in reporting this information in a security commit message?'' & Yes, No, Unsure\\\hline
        Q3 & ``Would you use this or a similar convention as standard practice in your own work or advocate its use in your team?'' & Yes, No, Unsure\\\hline
        Q4 & ``If you answered "Other" or "Unsure to any of the questions, please explain briefly below.'' & Open \\\hline
        Q5 & ``Please enter any other comments or suggestions below.'' & Open \\
    \hline
    \end{tabular}
    \caption{Survey questions used to validate SECOM.}
    \label{tab:survey}
\end{table}


\section{Findings}~\label{sec:findings}

In this section, we present the main conclusions 
of our empirical analysis of security commit messages 
and the validation of \texttt{SECOM}. 

\subsection*{\textit{RQ1: What information is included in
the commit messages of public security patches?}}


We ran our extraction pipeline (Figure~\ref{fig:parsing}) over a total of $11036$ security commit messages. Table~\ref{tab:entities} shows the number of entities extracted per entity category (\#Entities), the number of commits where entities of each type were found (\#Commits), and the respective percentage of commits (\%Commits). We divided the 
entities into two main sets: security-specific and commit-specific. The tool was able 
to extract key information for both types:
$38.5\%$ of the extracted entities are security-specific, and the remaining $61.5\%$ are commit-specific.

\begin{table}[t!]
\footnotesize
    \centering
        \caption{Extraction Results} 
    \begin{tabular}{| p{2cm} | p{1.25cm} | p{1.5cm} | p{1.5cm} | }
    \hline
        \textbf{Category} & \textbf{\#Entities} & \textbf{\#Commits} & \textbf{\%Commits} \\\hline
SECWORD & 16126 & 6749 & 61.2\%\\\hline
ACTION & 10364 & 6409 & 58.1\%\\\hline
EMAIL & 4738 & 2086 & 18.9\%\\\hline
SHA & 4943 & 1467 & 13.3\%\\\hline
FLAW & 4402 & 2843 & 25.8\%\\\hline
ISSUE & 3561 & 2805 & 25.4\%\\\hline
URL & 1175 & 929 & 8.4\%\\\hline
VULNID & 1799 & 1330 & 12.1\%\\\hline
VERSION & 658 & 571 & 5.2\%\\\hline
DETECTION & 629 & 374 & 3.4\%\\\hline
SEVERITY & 142 & 118 & 1.1\%\\\hline
CWEID & 25 & 23 & 0.2\%\\\hline\hline
\textbf{Total} & 48562 & 10168 & 92.1\%\\\hline
\end{tabular}
    \label{tab:entities}
\end{table}

\textbf{Commit-specific data (COM).} The tool extracted entities for all the $7$ different entity categories classified as commit-specific. A total of $29841$ entities were extracted as commit-specific information. 
A commit usually implies an \emph{action} or \emph{change}. In the case of security, it should imply fixing a vulnerability (corrective maintenance). Yet, only $10364$ entities for verbs (ACTION) were extracted from $6409$ ($58.1\%$) commit messages. Fixing a vulnerability also means fixing a flaw. Our tool extracted mentions of flaws (FLAW) in $2843$ ($25.8\%$) commit messages. 
Mentions of issues (ISSUE) and URLs (URL) can be a good source of external documentation and information, but security engineers rarely mention that information in their commit messages: only $2805$ contained references to issues, and $929$ commit messages contained URLs. 

% Previous work showed that only $38\%$ of security commit messages included security-relevant words~\cite{9678720}. 
\textbf{Security-specific data (SEC).} The tool extracted entities for all the $5$ different entity categories classified as commit-specific. A total of $16126$ security-related words (\texttt{SECWORD}) was collected from $6749$ out of $11036$ ($61.2\%$) commit messages.
Vulnerability IDs (VULNID) were only extracted for $1330$ commit messages, while all patches that integrate our dataset, fix a vulnerability with a known ID. Vulnerability IDs easily map security commits to official reports, which usually provide more details on the vulnerability, product affected, severity, and more. Therefore, they are important to add to commit messages.
It seems that security engineers rarely mention weakness IDs (CWEID)---only $25$ entities were extracted from $23$ commits messages. We suspect that we can often extract the weakness type by looking at the set of entities extracted for the SECWORD category. For instance, in ``fixed xss vulnerability bug by oncellhtmldata'' (commit message from security patch to the GHSA-hf4q-52x6-4p57 vulnerability), we could infer a CWE-79 based on the ``xss'' entity extracted in this message. Severity can be important for security patch management systems to know which security patches to prioritize. If a patch to a critical vulnerability is released, it should be installed as soon as possible. 
However, severity entities (SEVERITY) are rarely included in security commit messages---only extracted from $118$ ($1.1\%$) security commit messages.


\textbf{Finding 1.} Security engineers
use security-related words in
$61.2\%$ of the security commit messages
used to patch software vulnerabilities.



\textbf{Finding 2.} Vulnerability IDs, Weakness IDs and Severity are rarely mentioned 
in security commit messages---although important for manual and automated detection and prioritization. 

Our tool extracted
a total of $48562$ entities from $10168$ out of the $11036$ ($92.1\%$) security commit messages under analysis. For the remaining $868$ of the commit messages, we performed a manual validation of the reason behind of null extraction:
%(Table~\ref{tab:reasons}):

\textbf{Poorly-Written.} We found several types of poorly-written messages reported in previous work~\cite{Tian_2022}. For instance, we found $51$ messages containing only one token (``Single-word''). Some examples are ``update'', ``...'', ``:arrow\_up:'', ``Refactor''. Poorly-written messages usually contain one line without clear information regarding the commit’s purpose (e.g., ``applied updates'', ``backend media''. We found a total of $760$ messages that lacked meaningful information. 

\textbf{Non-Security Related.} Dense message with no clear relationship with security. One example is 
``Support progressive event for dc. This implements the progressive api event fo rthe dc image. This is currently only supported for vardct without extra channels: for
modular and extra channels it's only supported if squeeze is used, and
it may not correctly work with flush yet in that case.''---the commit message of OSV-2021-1606 vulnerability.
We found a total of $97$ commit messages that were hard to relate with security fixes. 

\textbf{Misspelling Issues.} Messages with misspelled english words that would be detected if well written. One example is the message: ``sanitzing user input ''.
However, this kind of issues reflects a very small percentage of the problem.



% \begin{table}[t!]
%     \centering
%         \caption{Reasons why the NER tool failed extraction} 
%     \begin{tabular}{| p{3cm} | p{4cm} | p{0.5cm} | }
%     \hline
%         \textbf{Pattern} & \textbf{Description} &  \textbf{No.}
%         \\\hline
%         Poorly Written  & Containing usually one line without clear information regarding the commit's purpose.   & 760 \\\hline
%         Non-Security Related  & Dense message with no relationship detected with security.  & 97 \\\hline
%         Misspelling Issues & Misspelled english words that lead to unsuccessful extraction. & 11 \\\hline
%         Total &  & 868 \\\hline
%     \end{tabular}
%     \label{tab:reasons}
% \end{table}

\textbf{Finding 3.} No extraction of entities was performed from $8\%$ of security commit messages mainly due to poorly written messages, misspelling issues and no clear connection with security.


\subsection*{\textit{RQ2.  Do security engineers follow best practices to write security commit
messages?}}

We applied $7$ compliance checkers to the commit messages to evaluate if security engineers are using generic best practices to write commit messages since no standard for security exists.

\textbf{C1.} One way of easily marking a commit message for an automated solution is with the usage of a prefix in the header (as the Conventional Commit Convention proposes~\cite{convcom}). In security, this practice was only used in 453 out of 11036 (4.10\%) of commits follow the conventional commits convention ``<type>(scope):'' using prefixes such as ``patch'' or ``fix''.

\textbf{C2.} Headers are important because they are ultimately used to make a quick searches of relevant commits through the git reflog feature. Therefore, it is important that all commit messages have a meaningful and clear header. All the security commit messages (100.00\%) have a one line subject/header. But only 4288 out of 11036 (38.85\%) headers have security-related words (SECWORD) and reflect an action (ACTION).

\textbf{C3.} Previous work has shown that a good commit message needs to clearly mention the problem (what), its impact (why) and how it will be fixed (how)~\cite{Tian_2022}. However, only 6875 out of 11475 (59.91\%) commit messages have a body. Security engineers do not use often security related-words (or vocab) in their body messages since only 4031 out of 11036 (36.53\%) of body messages have SECWORDS.

\textbf{C4.} Crediting the authors of the fixes by signing-off by commits is a very well known practice in security. However, only 925 out of 11036 (8.4\%) commit messages were signed-off by the authors.

\textbf{C5.} Reviewers are usually the internal developers or seniors developers that review and approve the issues. Only 368 out of 11036 (3.33\%) of messages have the reviewers contact.

\textbf{C6.} Issues are good sources of documentation since they sometimes provide details on the discussion of the problem and potential solution. However, only 2805 out of 11036 (25.42\%) commit messages have references to issues.

\textbf{C7.} Same as references to issues, links to bug trackers are also good to mention since they usually provide useful documentation on the problem, fix, severity, and more. But again, only 196 out of 11036 (1.78\%) of commits have references to bug trackers.


\textbf{Finding 4.} Security engineers, do not follow best practices to write security commit messages in general. Even when it seems they are, we concluded that key information is missing---which indicates we need best practices for writing better security commit messages. 


\subsection*{\textit{RQ3. How open is the security community to a new standard for security commit messages?}}

Feedback received from the security community suggests that they see value in SECOM and would like to see it evolve into a standard practice---6 out of the 9 participants responded ``Yes''
to Q3 (``Would you use this or a similar convention
as standard practice in your own work or advocate its use in your team?''), the remaining three participants answered ``Unsure''. None of the participants answered ``No''. 2 out of 9 participants said they did not find valuable to mention how the vulnerability was detected. However, 6 of the 9 participants answered ``yes''. Therefore, we considered the ``Detection'' field in the convention. The vulnerability ID, CWE ID,  and severity are the fields that participants found more important to include 
in the commit message. The least important fields are the CVSS and report link.

\textbf{Finding 5.} The security community sees value in SECOM and aims to adopt it as a standard practice in the future.

The participants that were unsure of SECOM's adoption were mainly concerned with the current practices: ``Folks won't be practicing this style daily, and for drive-by contributions, git messages are likely to be highly ad hoc and idiosyncratic.''. However, we argue that with good automated tools, we can help developers produce better commit messages for security. Writing more structured and informative commit messages is important since, ultimately, it can improve the detection and assessment capabilities of patch triage systems and enable fast patch management. 











\section{Implications and Considerations}\label{sec:discussion}

In this section, we describe the implications of 
our findings, ethical considerations, and how 
SECOM should be used.

\subsection{Dealing With Patch Transparency}\label{sec:transparency}

 Transparency is a double-edged sword. It can be a source of trust for consumers~\cite{Householder2020}, but it also creates vulnerability~\cite{9678720}, and, the need for security is often used to avoid transparency because of the risks that come from it. The reality is that non-transparent processes lead to abuse and make timely patch management a challenge (or even impossible). The CERT Coordinated Vulnerability Disclosure (CVD) guide suggests avoiding silent patches since it hinders public awareness of fixes to software vulnerabilities. This, then, leads to a lack of understanding and trust from developers and to poor triage systems (automated tools are incapable of reasoning about data and detect patches if no key information is provided).
 
 The SECOM convention and best practices purposed in this paper are meant to 
help vendors to produce better documentation and boost the patch 
management phase---responsible for deploying the new changes to the users. We are aware that being too much transparent can make users vulnerable to cyberattacks, but we argue that providing better documentation will help automated tools be more effective and fast for software security patch management. So, OSS users and companies can benefit from an automated solution by being aware 
of the fixes as soon as they are developed, which is usually one week earlier than
the disclosure~\cite{10.1145/3133956.3134072}.  

Experienced security engineers should determine when and not when to provide details. If development systems are private, 
then no major risks are attached to providing a detailed security commit message (in principle, we do not account for internal attackers). However, when a critical vulnerability is to be patched in open-source software of wide use, then security engineers should carefully decide which details to include in the commit message. They should provide enough information to users understand its criticality but not enough to attackers leverage.

One thing we do not advocate is the documentation of exploits since this would clearly benefit malicious attackers. That kind of information should be released later, after providing time for the users to deploy the patch.

\subsection{The Burden of a New Standard}

Applying new best practices is usually taken as a burden by the software and security communities, but it is also how we prevent and solve current challenges. As we mentioned before, well-structured and complete documentation is crucial to create trust between different parties and faster patch management---to decrease the user's exposure window. Therefore, we, as a team, and part of the software engineering community, are working on automated solutions for compliance validation and the generation of commit messages to reduce the bottleneck that our standard could introduce. \texttt{SECOM} can also be leveraged to guide other solutions to generate commit messages~\cite{7203049}. It is also important to notice that the open-source security community sees value in SECOM and aims to adopt it as a standard practice in the future. In fact, some security researchers are already using it and considering it as standard best practice~\cite{blackhat}, and more than 2k commit messages were produced following our solution.

\subsection{Need for Better Patch Documentation}

Previous work has shown that the software patch detection problem could not be solved only with security commit messages (or other types of metadata)~\cite{DBLP:journals/corr/abs-1806-05893,9678720}---which we agree. The solution proposed is to focus on code information instead. One study proposed a mixed solution~\cite{SSPatcher2022}. However, we think that focusing on code information will not fix the issue soon due to the complexity of the patterns under analysis (software vulnerability), the approaches gaps (e.g., machine learning, deep learning~\cite{9448435}), the diversity of weaknesses and many more. Instead, we can make an extra effort to provide better documentation and not only improve automated solutions but also build trust between vendors and users---which we argue will ultimately build a safer environment.


\section{Threats to Validity}\label{sec:t2v}

This section discusses the study's potential threats to validity.

\textbf{Internal Validity:}  In the survey study, our sample of participants is small, which may not reflect the entire population. However, we ensured that all the answers come from a reliable and expert source (OpenSSF). In addition, we only considered participants from the open-source community (i.e., private software developers may not be represented).

\textbf{External Validity:} We did not check for all the rules described in the standards considered. Therefore, our conclusion may not reflect the entire ground truth. But, we did check the most important rules, the ones related to having a type, header, and body in the commit message which improves the commit messages considerably.


% This section looks pretty short. Normally needs to be divided into External Validity (generalizability, related to the diversity of the data) and Internal Validity (possible flaws in the design, methodology, data collection, measurement, representation of constructs used). The one threat that was mentioned is internal validity, related to reliability. Is this section complete, or to be completed. At least one could say "the biggest internal threat to the study is ... (the one you mentioned)", and then comment on external validity, whether the conclusions can be generalized to all security commit messages, or a subset (e.g, open-source), or even a smaller subset of OS typified by the data collected.


\section{Related Work}\label{sec:rw}

This section summarizes previous research work in the fields
of Software Security Patch Management, Software Security Patch
Detection and Commit Message Analysis.

\subsection{Software Security Patch Management}

Software security patch management is the process of identifying, acquiring,
testing, installing, and verifying security patches for software products and systems~\cite{DISSANAYAKE2022106771}. 
Security patches are considered the most effective strategy to mitigate software vulnerabilities~\cite{DISSANAYAKE2022106771,SOFT-PATCH-MANAG-NIST,10.1145/2660267.2660329}.
They are prioritized over non-security patches as they aim to protect users from cyberattacks~\cite{SOFT-PATCH-MANAG-NIST}. 
Yet, one important challenge still prevails, the \emph{lack of efficient patch triage systems} to identify and prioritize security patches~\cite{Zhang2021AnIO, SSPatcher2022,hacker-news-patches,DBLP:conf/soups/LiRMMC19}: current processes are largely manual (time-consuming) and prone to ignore important bug fixes such as the one behind Equifax~\cite{failed-to-deploy-patch}.
Figure~\ref{fig:cycle} shows part of the patch management process: \emph{1)} vendors release a patch to mitigate a vulnerability previously identified or reported; \emph{2)} automated or manual systems identify and prioritize the patch and trigger patch deployment; \emph{3)} the patch is deployed successfully into the users' machines. In this paper, we propose a solution that can improve the understanding and identification of security patches for automated and manual triage systems, which can consequently decrease the time users are exposed to vulnerabilities. 
Silent fixes, i.e., software vulnerabilities patched without any release note, hinder security patch management since they leave users unaware of the new patches and their criticality~\cite{Householder2020}---knowledge about the fix is key for successful patch deployment. 
Previous work has shown that $56.7\%$ of the security commit messages are documented poorly~\cite{10.1145/3593434.3593481}, which hinders the different triage systems tasks.
This paper presents a solution to make security commit messages more informative and enable the effectiveness of patch management processes.

\subsection{Software Security Patch Detection}

Version control systems are used to maintain a record of code changes
and manage access to codebase development artifacts~\cite{10.1145/2950290.2950364}. As with any type of code changes, security patches are pushed to codebases through commits. Each commit contains changes to source code and a message explaining what changes
are made and why~\cite{883028,Tian_2022}. Over the past years, many approaches have emerged to detect security patches through metadata (e.g., commit logs, commit metadata, and associated reports) and code changes. Here, we only
mention approaches considering commit logs (or messages).
\emph{Reis et al.} detected security patches for 16 types of weaknesses in commit messages by applying different regular expressions per each kind of weakness~\cite{reis2017secbench}.
% Researchers from the security industry 
% % (Zhou and Sharma 2017; Sabetta and Bezzi
% % 2018) (from SourceClear, Inc., and SAP, respectively)
% have presented early investigations on the prediction of security issues in relation with
% commit changes.
\emph{Zhou and Asankhaya}  leveraged regular expressions to identify
features for predicting security-relevant commits in commit logs,
commit metadata, and associated bug reports. Features are vectorized and learned by word2vec~\cite{10.1145/3106237.3117771}. 
\emph{Sabetta and Bezzi} extended previous work by considering code changes as well~\cite{DBLP:journals/corr/abs-1807-02458}---which led to considerable performance improvements.
\emph{Sawadogo et al.} proposed SSPCatcher, a co-Training-based approach to catch
security patches as part of an automatic monitoring service of code repositories~\cite{SSPatcher2022}. Detection is based on the result of two Support Vector Machine classifiers: 1) one trained with commit logs; and, 2) one trained with code features. In order to identify a new security patch, both 
classifiers have to be in agreement. In this study, we do not propose a new model or approach to detect security patches. Instead, we empirically analyze security commit messages and propose a standard that can improve the effectiveness of the different triage systems tasks (e.g., detection, assessment, and prioritization).

\subsection{Security Commit Message Analysis and Standards}

Previous work has shown that only $38\%$ of security commit messages used to ``silently'' patch software vulnerabilities in the past included security-related words~\cite{9678720}. In this work, 
we analyzed the commit messages attached to all security patches to known software vulnerabilities instead of only focusing on silent fixes. As mentioned before, we were unable 
to find a convention or standard for security commit messages. Instead, we only 
found guidelines and specifications for general commit messages. The preliminary results
of this study were published before~\cite{9796324}. This paper is an extension and 
detailed report on how we analyzed the data, the results, and the design of our solution.

\section{Conclusions and Future Work}\label{sec:fw}

In this study, we observed that security commit messages integrate some relevant information for automated and manual triage systems but not enough to map them to bug reports or know which patches to prioritize. We also observed that no standards exist for security and that security engineers usually do not follow best practices. This paper presents a solution for 
the lack of structure, and key information in the  
commit messages of security fixes. we built SECOM, a convention for security commit messages, based on our empirical analysis findings and the open-source security community feedback and approval. 

Although part of the security community worries about the attacks that potentially will come from transparency, there seems to be space in the security community for a new standard that is already helping security engineers produce better security commit messages. However, it should be used carefully and consider the software vulnerability impact and time response of the maintenance team. 
%
In the future, we plan to release automated solutions for compliance validation and message generation to reduce the burden that a new best practice can introduce in the patching and disclosure life-cycle. 

% \section{Section B}
\label{sec:sectionb}

\subsection{Subsection A}
\label{subsec:subasectionB}


\cleardoublepage

% %%%%%%%%%%%%%%%%%%%%%%%%%%%%%%%%%%%%%%%%%%%%%%%%%%%%%%%%%%%%%%%%%%%%%%
% Dummy Chapter:
% %%%%%%%%%%%%%%%%%%%%%%%%%%%%%%%%%%%%%%%%%%%%%%%%%%%%%%%%%%%%%%%%%%%%%%

% %%%%%%%%%%%%%%%%%%%%%%%%%%%%%%%%%%%%%%%%%%%%%%%%%%%%%%%%%%%%%%%%%%%%%%
% The Introduction:
% %%%%%%%%%%%%%%%%%%%%%%%%%%%%%%%%%%%%%%%%%%%%%%%%%%%%%%%%%%%%%%%%%%%%%%
\fancychapter{Infrastructure-as-Code Scripts Application}
\label{cap:chapter}

\textit{This chapter studies and improves the precision and recall of IaC tools through practitioners feedback and Generative Artificial Intelligence.}

\input{4.IAC/1.Introduction.tex}
\section{Leveraging Practitoners Feedback to Improve Precision}
\label{sec:intro}

Software configuration management and deployment tools like
Puppet\footnote{\url{https://puppet.com/}}, 
Ansible\footnote{\url{https://www.ansible.com/}},
and Chef\footnote{\url{https://www.chef.io/}} became 
popular amongst software development warehouses~\cite{8919181,rahman2017factors}. 
The adoption of these tools has increased with the growing 
movement to run software in cloud servers. These tools help 
infrastructure teams increase productivity by automating various 
configuration tasks (\myeg{}, server setup). 
In short, these tools describe the environment configuration in 
a set of provisioning scripts that can be versioned and 
reused. They enable configuration consistency between different 
environments and can reduce the time required to provision and 
scale the infrastructure.
The process of managing and provisioning infrastructure through 
configuration scripts is called Infrastructure-as-Code (\iac). 
\iac\ tools take a script as input and create an infrastructure 
that typically runs in a virtual environment as output.

%
As with any piece of code, IaC scripts are also prone to defects 
such as security vulnerabilities~\cite{8812041}. For example, 
in $2020$, Palo Alto Network researchers reported the discovery 
of over $199$K vulnerable IaC templates~\cite{paloalto-unit42}. 
Specifically, $42\%$ of AWS CloudFormation templates, $22\%$ of 
Terraform templates, and $9\%$ of Google Kubernetes YAML files 
were vulnerable. In addition, researchers found more than $67$k 
\textit{potential} security smells in \iac\ scripts implemented 
in Ansible, Chef, and Puppet~\cite{9388795} through an ad-hoc 
tool created to show the presence of a new set of anti-patterns 
for security in the \iac{} domain. 
These reports highlight the importance of tools to prevent 
vulnerabilities from reaching production and shift security left 
in the development pipeline. 

%
Figure~\ref{lst:manifest-example} shows an example of an
\textit{Admin by default} weakness (CWE-250\footnote{CWE-250 details
available at \url{https://cwe.mitre.org/data/definitions/250.html}}),
a potential vulnerability in a Puppet manifest in a module of the
\texttt{PuppetLabs}.\footnote{Admin by default example available at 
\url{https://github.com/puppetlabs/puppet\_operational\_dashboards/blob/9eb67a407aa44c2f924f67f207edc7032f81f86a/manifests/profile/dashboards.pp\#L137} (Accessed \today)}
The vulnerability manifests when the developer
configures a user as ``admin'' or ``root'' for an
infrastructure component. In this example, the \texttt{\$grafana\_user}
is set as ``admin'' for the different services (Puppetserver, Puppetdb, Postgresql, Filesync) used by Grafana\footnote{Grafana is an open-source software
application for data exploration and visualization.
More information available at \url{https://grafana.com/}}. Therefore, 
any service can be prone to a privilege escalation attack.
In \iac{}, all the infrastructure components
are configured through scripts, including the access credentials.
Specifying default users as administrative gives privileges to users
that only an administrator should have. Admin accounts can be exploited 
to access sensitive data and execute unauthorized
code/commands. Infrastructure engineers should avoid setting admin
passwords and usernames to user accounts that do not need the privileges.
Detecting these issues automatically essential to 
make infrastructure engineers aware of possible security problems and
to help companies build more robust infrastructures.


\begin{figure}[t!]
  \centering
  \includegraphics[width=\linewidth]{Figures/Figure-1.pdf}
  \caption{Simplified example of an \textit{Admin by default}.}\label{lst:manifest-example}
  \vspace{-3ex}
\end{figure}


\vspace{1ex}\noindent\textbf{Problem: Puppet IaC Security Linters are not reliable yet!}~In $2019$, Rahman~\etal{}
showed that \iac\ scripts---just like any piece of code---are not immune to security
vulnerabilities~\cite{8812041,10.1145/3408897}. They focused on Puppet configuration files and listed seven
anti-patterns that could lead to security vulnerabilities. The work led to the development of \slic,
a linter to detect those defects in Puppet scripts. Linters are often
imprecise tools~\cite{6606613,7781843,10.1145/1646353.1646374,8622456,8530713,46576}.
Therefore, motivated by the report of very high accuracy (i.e., precision and recall)
from their paper (Table IV~\cite{8812041}), we
decided to conduct a reproduction study of \slic{} based on a
different and larger set of projects. We asked students (co-authors of
this paper) and developers to analyze the warnings that the tool reports.
To validate the students observations of low precision, we reported a sample of the
warnings of the tool to maintainers of \botTotalRepos\ open-source projects. From
the \botTotalIssues\ issues created, we obtained responses to
\botTotalIssuesAnswers\ issues where only \botFinalIssues\ issues were discussed or clear. 
Results showed that the tool performs differently in a new 
set of projects, particularly when validated by the software owners. 
The precision observed was 
smaller than the one reported in SLIC's original 
paper (\botPrecision\ instead of $99\%$)
which indicates that security IaC linters for Puppet
are not reliable yet due to the high false positive rates.

Like many linters, \slic\ uses simple rules to detect issues.
Essentially, it searches for string patterns in
the values of tokens (many times) regardless of their 
type (e.g., variable, string, etc.)
and the relationship between them. For instance, the
``Usage of Crypto. Algorithms'' checker  
(CWE-326\footnote{CWE-326 details are available at \url{https://cwe.mitre.org/data/definitions/326.html}})
searches for any token whose value includes \texttt{sha1} or \texttt{md5}.
Both are built-in Puppet functions and \slic{} fails to consider
the context of usage of these algorithms, i.e.,
these functions are called in Puppet manifests to
encrypt data (e.g., \CodeIn{$encrypt\_key = md5($key)}).
Therefore, \slic{} incorrectly detects 
\CodeIn{md5checksum = '07bd73571b7028b73fc8ed19bc85226d'} as a CWE-326. 
This simplicity creates much noise for developers. In this preliminary study, 
we observed that the rules for the current IaC security anti-patterns 
must be better designed to be safely adopted by the industry and avoid 
productivity disruption.

\vspace{1ex}\noindent\textbf{Solution:}~Our preliminary study revealed
that (1) there is a need to improve the precision of IaC security linters for Puppet, 
and that (2) security tools can be iteratively improved and extended by incorporating 
feedback from the developer community as suggested in previous work~\cite{46576}. 
This paper reports on 
the process we followed to iteratively and 
incrementally improve the precision of an IaC linter according to user 
feedback. For example, the experiments described above ignited 
discussions with members of the development and security teams of 
Puppetlabs\footnote{GitHub PuppetLabs
organization website: \url{https://github.com/puppetlabs}}, as well as one
project manager from Vox Pupuli\footnote{Vox Pupuli is the organization responsible for maintaining
modules and tools for the Puppet community: \url{https://voxpupuli.org/}}.
The feedback collected from the team, OSS maintainers and the Puppet 
community led to the creation of a new tool, which we dubbed as \toolname{}.
Later, we leveraged the expertise of practitioners experienced 
in IaC tools or security to iteratively and incrementally
improve the new tool.

Figure~\ref{fig:timeline} shows the timeline of feedback collection 
followed to design and improve \toolname{}.
To sum up, we bootstrapped the design of \toolname\ with rules obtained 
from the revision of SLIC's ruleset, according to the feedback of the 
research team and owners of OSS projects (\textit{phase 1} in
Figure~\ref{fig:timeline}); and, incrementally evolved the linter
according to the recommendations of practitioners
(\textit{phase 2} in Figure~\ref{fig:timeline}). We improved
\noRulesSlic\ rules of the \slic{} ruleset and added
\newRules\ new rules that were either recommended by practitioners (e.g. weak
password); or relevant for the infrastructure domain (e.g. homograph
attacks\footnote{Apple Domain Attack (2017):
\url{https://www.xudongz.com/blog/2017/idn-phishing/}} and malicious dependencies).

\begin{figure}[t!]
  \centering
  \includegraphics[width=\linewidth]{timeline.png}
  \caption{Timeline of feedback collection.}\label{fig:timeline}
  \vspace{-3ex}
\end{figure}

\vspace{1ex}\noindent\textbf{Main Results:}
This paper performs the following contributions:

\begin{itemize}[topsep=.2ex,itemsep=.2ex,leftmargin=0.8em]
    \item[\Contrib{}]\textbf{Study.}~A replication study of \texttt{SLIC}'s precision,
    including a preliminary study conducted with two researchers (co-authors of this paper)
    and a study with several \github\ scripts validated by project maintainers;
    \item[\Contrib{}]\textbf{InfraSecure.}~A new linter
    adjusting \noRulesSlic\ rules of the original
    \slic\ ruleset and adding \newRules\ new rules with a final precision of \finalPrecision.
    \item[\Contrib{}]\textbf{Dataset.}~A dataset of \iac{} scripts with more than $1000$ warnings
    classified as false positives and true positives that researchers can use to evaluate
    other security linters;
\end{itemize}

\noindent\textbf{\textit{Take-away message:}}~The takeaway message of
this paper is that it is feasible to tune security linters to produce
acceptable precision for important classes of warnings (confirming the
findings reported in a study at Google~\cite{46576}) and
that involving practitioners in discussions is an effective way to
guide the improvement of those linters.

\noindent\textbf{\textit{Replication Package:}}~All the scripts and data
    used in this study (including feedback obtained from the maintainers and practitioners)
    are available at: \repPackage{}.

	%%%%%%%%%%%%%%%%%%%%%%%%%%%%%%%%%%%%%%%%%%%%%%%%%
\section{Background}\label{sec:background}
%%%%%%%%%%%%%%%%%%%%%%%%%%%%%%%%%%%%%%%%%%%%%%%%%

This section provides background information on Puppet and
discusses security weaknesses in \iac\ scripts.

\begin{table}[!t]
  \small
  \centering
  \setlength{\tabcolsep}{3pt}
  \caption{Examples of security smells per weakness.}
  \vspace{-2ex}  
  \label{tab:examples}
  \begin{tabular}{ll}
   \toprule
   \textbf{CWE}   &  \textbf{Example} \\\midrule
   CWE-798  & \$username = ``mariadb''\\
   CWE-259  & \$password = ``!TQ23Rg''\\
   CWE-321  & \$key = ``A67ANBD7''\\
   CWE-319  & \$req = ``http://www.domain.org/secret'' \\
   CWE-546  & \#https://bugs.debian.org/cgi-bin/bugreport.cgi?bug=538392 \\
   CWE-326  & password => md5(\$debian\_password) \\
   CWE-284  & \$bind\_host = ``0.0.0.0'' \\
   CWE-258  & \$rabbitmq\_pwd = ``'' \\
   CWE-250  & \$user = ``admin'' \\
   CWE-521  & pwd => ``12345'' \\
   CWE-1007 & \$source = "http://deb.debi\mhl{a}n.org/debi\mhl{a}n" \\
   CWE-829  & \$postgresql\_version = 8.4 \\
   \bottomrule
  \end{tabular}
  \vspace{-2ex}
\end{table}

% %
\subsection{Puppet}

Developers no longer
deal with systems composed of a single machine and database.
Instead, system administrators must manage multiple diverse operating systems, databases, and virtual machines.
Perhaps most importantly, they must ensure their
configurations are consistent at any given time. Configuration management
technologies have been around for over a decade---Puppet was founded in $2005$.
Puppet is a solution that helps IT infrastructure management through 
code by supporting software deployment, packages,
and configuration. In Puppet, programs
are written using Puppet's declarative language. They are
called manifests. More information about the language can be found 
elsewhere\footnote{More about Puppet: https://puppet.com/docs/puppet/7/puppet\_overview.html}.


\subsection{Security Weaknesses}

This section describes in more detail the potential weaknesses 
in Puppet scripts. Table~\ref{tab:examples} illustrates 
examples of each weakness.

\textbf{Hard-coded secrets (CWE-798):} This warning refers to the practice of
including sensitive information such as passwords or cryptographic keys in
code files. Table~\ref{tab:examples} shows a hard-coded 
password example.
Rahman \etal{} considered $3$ kinds of data as
sensitive: usernames, passwords, and cryptographic keys. 
However,
the Common Weakness Enumeration\footnote{CWE-798 description is available at \url{https://cwe.mitre.org/data/definitions/798.html}}
list does not consider solo hard-coded usernames as a threat.
Practitioners
involved in our validation studies shared the same opinion. Therefore,
we argue that hard-coded usernames should be only reported
when there is a paired password (CWE-259) or cryptographic key (CWE-321).
More discussion on this is provided later in Section~\ref{sec:design}.


\textbf{Use of HTTP without TLS (CWE-319):} This warning refers to the practice
of using HTTP without the Transport Layer Security (TLS) to transmit
sensitive data. This means that attackers can more efficiently exploit the
communication channel as the data is transmitted unencrypted, as
cleartext\footnote{This type of issues can lead to man in the middle (MITM) attacks: \url{https://owasp.org/www-community/attacks/Man-in-the-middle_attack}}.

\textbf{Suspicious comment (CWE-546):} This warning refers to
 comments that suggest the presence of bugs, missing security functionalities,
or weaknesses of a system. Details provided in comments about bugs, security functionalities
or weaknesses can be crucial for hackers to exploit the infrastructure.

\textbf{Use of weak cryptography algorithms (CWE-326):} This warning refers to
using weak crypto algorithms.
Attackers can easily crack the encryption schemes and have
access to the data~\cite{10.1007/11426639_2}.

\textbf{Invalid IP address binding (CWE-284):} This warning refers to
assigning the address 0.0.0.0 to a server.
This practice allows connections from any IP address to access 
that server~\cite{DBLP:conf/raid/Mutaf99}.
While mail servers have to listen on 0.0.0.0 to receive
mail, database servers should not since it can lead
to critical data breaches.

\textbf{Empty password (CWE-258):} This warning refers to using
empty strings as passwords, which are easily
guessable.

\textbf{Admin by default (CWE-250):} As detailed in the Introduction, this warning refers to
defining users with administrative privileges. This can result in security
weaknesses since it can disable or bypass security checks performed by the system.\footnote{Why
you should not use an admin account: \url{https://www.lbmc.com/blog/why-you-should-not-use-an-admin-account/}}

\textbf{Weak Password (CWE-521):}
This warning refers to the usage of weak passwords. Weak passwords
are easily guessable and can be bypassed to gain access to systems.

\textbf{Insufficient Visual Distinction of Homoglyphs Presented to User (CWE-1007):}
This warning refers to malicious actors using homoglyphs
that may cause the user to misinterpret a glyph and perform an unintended, insecure action.
The homograph attack performed against the apple website\footnote{Phishing
with Unicode domains:~\url{https://www.xudongz.com/blog/2017/idn-phishing/}} is a well-known
example of this type of weakness. Table~\ref{tab:examples} shows an example of a domain where 
the character ``\mhl{a}'' could be replaced by the respective homoglyph, as in the apple attack.
This warning might be essential to uncover malicious domains implanted by malicious open-source
contributors---typosquatting attacks.\footnote{OpenSSF post on scanning OSS software for malicious behavior:~\url{https://openssf.org/blog/2022/04/28/introducing-package-analysis-scanning-open-source-packages-for-malicious-behavior/}}

\textbf{Malicious Dependencies (CWE-829):} This warning refers to malicious software
by nature, i.e., dependencies that integrate known vulnerabilities (CVEs).
These are the leading cause of supply chain attacks, and one of the main
challenges the security community faces nowadays~\cite{9402108}. 

\section{Preliminary Study}\label{sec:preliminary_study}

This section reports on the findings of two studies---involving
different sets of participants---to assess the performance of SLIC, 
a recently-developed linter for Puppet.


\subsection{Validation with Students}\label{sec:research_team}

This section reports on a study involving two of this paper's 
authors to assess the precision of \slic\ on an
independent benchmark. The study consisted in inspecting
\fpSampleSize{} warnings reported by the tool. The warnings
were validated by one senior PhD student whose research focuses 
on security and static analysis and one junior PhD student 
in software engineering with basic security skills. 

\subsubsection{Dataset}\label{subsec:dataset}

\sloppy
To build our dataset, we mined \github\ projects containing \texttt{Puppet} scripts. 
We used three different queries to search for repositories: 1) \CodeIn{language:puppet is:public}; 2) 
\CodeIn{puppet in:readme is:public}; and, 3) \CodeIn{devops is:public}. We discarded results pointing 
to forked repositories (to avoid duplicates) and discarded results pointing to repositories without 
any code in \texttt{Puppet} scripts. Our crawler found a total of \totalMinedRepos{} GitHub repositories 
and \totalMinedScripts{} associated \texttt{Puppet} scripts. \slic\ scanned these scripts for the
seven sins and reported a total of \textbf{\minedWarnings} security warnings involving 
\minedScriptsWarnings{} \texttt{Puppet} scripts (=$26.5$\% of the total) from \minedReposWarnings{} 
repositories (=$73.5$\% of the total).
Table~\ref{tab:sins_large} shows the breakdown of warnings reported by \slic. Column ``Rule'' shows the name (kind) of the warning, 
column ``\#'' shows the number of warning reports of that kind, and column ``\%'' 
shows the percentage of the total associated with that number. 
This table lists the warnings in order of their prevalence.

  \begin{table}[t]
    \small
    \centering
    \setlength{\tabcolsep}{10pt}
    \caption{\label{tab:sins_large}Breakdown of warnings reported by \slic{}.}
    \vspace{-2ex}
    \begin{tabular}{lrr}
      \toprule
      Rule & \multicolumn{1}{c}{\#} & \multicolumn{1}{c}{\%}  \\
      \midrule
      Hard-coded secrets & \hardcodedSecretsMined{} & 69.9 \\
      Use of HTTP without TLS & \httpWithoutTLSMined{} & 11.7 \\
      Suspicious comments & \suspiciousCommentsMined{} & 8.7 \\
      Use of Weak Crypto. Algos. & \weakCryptoMined{} & 4.7\\
      Invalid IP Address Binding & \emptyPassMined{} & 2.4\\
      Empty Password & \invalidIPMined{} & 2.1\\
      Admin by default & \adminDefaultMined{} & 0.5 \\
      \midrule
      Total & \minedWarnings{} & 100 \\
      \bottomrule 
    \end{tabular}
    \vspace{-2ex}    
  \end{table}


\subsubsection{Methodology}\label{sec:pre_meth} \textbf{Samples.}
Given the high number of warnings reported by the tool (\minedWarnings) and the need for 
humans to analyze each warning, we sampled a set of reported warnings. 
We leveraged two popular complementary sampling strategies to that end~\cite{strat-sampling}. 
\textit{Stratified sampling} is a method to draw samples from a set by taking into account the 
distribution of kinds---in our case, the distribution of kinds of warnings. A \textit{proportional} (stratified) sampler draws samples in a number proportional to 
the size of the sets associated with each kind whereas an \textit{uniform} (stratified) 
sampler draws the same number of samples for each kind.
% To contrast, a proportional sampler 
% draws much more warnings of the kind ``Hard-coded secrets'' compared to ``Admin by default''
% whereas an uniform sampler draws the same number of warnings for both types. 
Intuitively, 
a proportional sampler values more the most prevalent kinds of warnings (as to make more 
accurate measurements on those kinds) whereas an uniform sampler treats every kind equally 
(as to avoid inaccurate measurements on uncommon kinds).
We sampled \proportionalSampleSize{} warnings \textit{proportionally} and \uniformSampleSize{} 
(=36*7) warnings \textit{uniformly}. In total, we analyzed  
\fpSampleSize\ warnings, which is a substantial increase when compared to the 
\akondFpWarnings\ warnings analyzed in the \slic{}'s paper~\cite{10.1145/3408897}. \textbf{Metric.}
We focused on precision to measure the reliability of the reports of 
the tool. 
% Precision is the ratio between the number of true warnings 
% (True positives or TP) reported and the total number of warnings reported by a 
% given tool, including false warnings (false positives or FP). 
% \begin{align}
% 	Precision = \nicefrac{TP}{TP+FP}\label{eq:prec} 
% \end{align}
Precision is especially important for security linters. Reporting scores of false 
warnings can be highly disruptive for a team's productivity, as team members tend 
to interrupt work to address high-priority tasks~\cite{DistefanoEtAlCACM2019}. Developers 
are less willing to use linters with low rates of precision because they find them not 
trustworthy and unreliable~\cite{46576}. \textbf{Statistical Tests.} Each one of the \fpSampleSize\ warnings was 
manually inspected by two co-authors. Cases where the authors found disagreement were discussed to reach 
a consensus. We report on the results of a Cohen's Kappa
analysis~\cite{cohen1960coefficient} to measure the inter-rater 
reliability of human decisions.
%
\subsubsection{Results}

\begin{table}[t]
  \small
  \centering
  \setlength{\tabcolsep}{3.5pt}
  \caption{\label{tab:prel_analysis_slic}Performance of SLIC. (Validation with Students)}
  \vspace{-2ex}
  \begin{tabular}{lrrrrrr} 
    \toprule
    \cellcolor{Gray} \textbf{\slic{}} & \multicolumn{3}{c}{\textit{proportional}} & \multicolumn{3}{c}{\textit{uniform}} \\\midrule
    Rule & \#TP & \#FP & Pr. &  \#TP & \#FP & Pr. \\
    \midrule
    Hard-coded secrets & \tpHardcodedSecretsProportional{} & \fpHardcodedSecretsProportional{} & \precHardcodedSecretsProportional{} & \tpHardcodedSecretsUniform{} & \fpHardcodedSecretsUniform{} & \precHardcodedSecretsUniform{} \\
    Use of HTTP without TLS & \tpHttpWithoutTLSProportional{} & \fpHttpWithoutTLSProportional{} & \precHttpWithoutTLSProportional{} & \tpHttpWithoutTLSUniform{} & \fpHttpWithoutTLSUniform{} & \precHttpWithoutTLSUniform{} \\
    Suspicious comments & \tpSuspiciousCommentsProportional{} & \fpSuspiciousCommentsProportional{} & \precSuspiciousCommentsProportional{} & \tpSuspiciousCommentsUniform{} & \fpSuspiciousCommentsUniform{} & \precSuspiciousCommentsUniform{} \\
    Use of Weak Crypto. Algorithms & \tpWeakCryptoProportional{} & \fpWeakCryptoProportional{} & \precWeakCryptoProportional{} & \tpWeakCryptoUniform{} & \fpWeakCryptoUniform{} & \precWeakCryptoUniform{} \\
    Invalid IP Address Binding & \tpInvalidIPProportional{} & \fpInvalidIPProportional{} & \precInvalidIPProportional{} & \tpInvalidIPUniform{} & \fpInvalidIPUniform{} & \precInvalidIPUniform{} \\
    Empty Password & \tpEmptyPassProportional{} & \fpEmptyPassProportional{} & \precEmptyPassProportional{} & \tpEmptyPassUniform{} & \fpEmptyPassUniform{} & \precEmptyPassUniform{}\\
    Admin by default & \tpAdminDefaultProportional{} & \fpAdminDefaultProportional{} & \precAdminDefaultProportional{} & \tpAdminDefaultUniform{} & \fpAdminDefaultUniform{} & \precAdminDefaultUniform{} \\
    \midrule
    Total & \tpProportionalSample{} & \fpProportionalSample{} & \precTotalProportional{} & \tpUniformSample{} & \fpUniformSample{} & \precTotalUniform{} \\
    \bottomrule 
  \end{tabular}
  \vspace{-5ex}
\end{table}


Table~\ref{tab:prel_analysis_slic} shows \slic{}'s results for both 
sampling strategies:
\textit{proportional} and \textit{uniform}. 
For each sampling strategy, we present the number of true positives~(\#TP), 
the number of false positives~(\#FP), 
and the Precision. Considering the results for \textit{proportional} 
sampling, the authors found a total of \tpProportionalSample{} 
true positives and \fpProportionalSample{} false positives. 
The average precision of \slic\ was \precTotalProportional{} for the 
proportional set. Considering the results for \textit{uniform} sampling, 
the authors found a total of \tpUniformSample{} true 
positives and \fpUniformSample{} false positives. 
On average, \slic's precision was \precTotalUniform{} for the uniform set. 

We ran a Cohen's Kappa analysis to measure the inter-rater reliability of human 
decisions in classifying the warnings. The kappa coefficient ($k$) shows the level 
of agreement between the two co-authors. The analyses yielded $k$=$0.89$ 
and $k$=$0.94$ for the \textit{proportional} and the \textit{uniform} sampling 
sets, respectively. According to McHugh's interpretation of $k$~\cite{mchugh2012interrater}, 
the reported levels of agreement 
are strong and almost perfect for the proportional and uniform sampling sets, respectively. 
For illustration, agreement was reached in $482$ out of $502$ warnings. 
The two co-authors discussed the warnings that 
raised disagreement. Cases where 
consensus was not reached were replaced by a new one and re-evaluated.
Cases where agreement was reached were updated with the final 
conclusion---inferred from the discussion between both co-authors. 

\textit{\textbf{Summary:}} Results show that the original precision of \slic\ drops 
considerably from $99\%$ (reported in the original work~\cite{8812041}) to below $65\%$ 
when tested in a new set of puppet scripts---which might indicate that \slic\ needs to 
be improved. However, the lack of context on the software under analysis by the co-authors 
may also be the reason for a lower precision. Therefore, we conducted a new experiment
with the owners and maintainers of open-source software, i.e., people with more knowledge 
and context of the applications.



\subsection{Validation with Owners of OSS projects}\label{subsec:maintainers}

\sloppy
Complementing the preliminary study reported in the previous section,
this section reports on the findings of a validation study of
\slic\ conducted with the maintainers of open source projects. The
motivation of this study was to confirm the observations of the 
previous experiment but now with open source developers, i.e., 
developers with more context of the software under analysis.  
\github\ issues 
were designed to
illustrate the security smells (including references to the corresponding
CWEs\footnote{Common Weakness Enumeration (CWE) taxonomy available at
\url{http://cwe.mitre.org}.}) and to guide the developer towards
patching the issue. We followed guidelines for issue reporting from
the literature. Issues include code samples, links to more
information and we strive to make the report message as brief and
objective~\cite{carvalho2020c}. Figure~\ref{fig:issue} shows an
example of an issue created for a ``Hard-coded Secret''. The title is
``Potential vulnerability in Puppet file: Hard-coded Secret''. The
issue (1) shows where the potential vulnerability is (code:
\CodeIn{cron\_user = 'root'}, script:
\CodeIn{puppet-apt\_mirror/manifests/init.pp} and line: $191$); (2)
explains the vulnerability and its possible implications (bypass
protection mechanism, gain privileges on applications, and access to
sensitive data); and (3) makes a recommendation to the developer on
how she can fix the vulnerability (by using a vault, in this
case). We created different templates of this message for the
different warning types.

\begin{figure}[t!]
  \centering
  \includegraphics[width=\linewidth]{issue.png}
  \vspace{-4ex}  
  \caption{Example of issue opened (based on SLIC report).}\label{fig:issue}
  \vspace{-3ex}
\end{figure}
%

%
\subsubsection{Dataset}

The dataset used in this study was different from the one from 
Section~\ref{sec:pre_meth}. 
We mined \github\ projects with activity in $2020$ 
(at least one commit) and containing Puppet scripts. 
We conjectured that focusing on projects with recent activity would increase 
our chances of obtaining responses. We used two different queries to search for 
repositories: 1) \CodeIn{language:puppet is:public}; and, 2) \CodeIn{puppet in:readme is:public}. 
We discarded results pointing to forked repositories (to avoid
duplicates), repositories without any activity in $2020$ (no commits),
and repositories without any code in Puppet per project. Our tool
scanned \botTotalScripts{} Puppet scripts from \botWarningsRepos{}
GitHub repositories. In total, \botTotalWarnings{} security warnings
were detected in \botScriptsWarnings{} Puppet scripts (=$30.7\%$ of
the total). We issued alerts to projects with maintainers involved 
in the slack of the Puppet community. We received \botTotalIssuesAnswers{} answers to 
the \botTotalIssues{} issues we submitted, but only \botFinalIssues{}
issues were clearly validated by practitioners---which were 
the ones we considered for our conclusions.

\subsubsection{Methodology} This section presents the methodology 
followed to submit the issues. \textbf{Sample.} A total of
\botWarningsRepos{} \github\ repositories were scanned to this
study. We ensured that the repositories had recent activity (at least
one commit in $2020$) to improve the probability of obtaining
responses. The total amount of scripts scanned was
\botTotalScripts{}
---$7$ times the sample used in our preliminary
study (Section~\ref{subsec:dataset}) and $28$ times the sample used in
Rahman et. al~\cite{8812041} to evaluate the SLIC's precision and
recall. 
\textbf{Issues.} We reached out to the software owners 
through the Puppet community slack and submitted issues to projects 
with maintainers involved in the slack community. All the issues followed
a specific template depending on the type of warning
(cf. Figure~\ref{fig:issue}). The issued message not only located and
explained the vulnerability but also recommended an example of a
patch (i.e., actionable messages to save maintainers time). \textbf{Reply Evaluation.} We monitored the discussion threads
associated with the issues. For each response obtained, we classified
the warnings reported as true positives (TP) or false positives (FP)
according to the opinion of the maintainers in their responses.
Issues closed by the maintainers without any reply or activity were discarded.
Issues closed with unclear responses (e.g., 
``N/A'', ``:thumbs\_down'') were also discarded since they did not provide
clarity on the validation of the issue. We only considered the issues where there was some sort 
of discussion of the issue (e.g., ``These todo's shouldn't be there, I agree ... but it's not about defects/weaknesses here. It's 
just a marker to include more operating systems in the future.'') or a clear validation of the issue (e.g., ``All false positives'', ``This is not a secret.'').
From the answers
obtained, two of the co-authors manually inferred the classification
of each warning. We use Cohen's Kappa~\cite{cohen1960coefficient} to
measure the inter-rater reliability of human
decisions. \textbf{Metrics.} We measured Precision as
described in Section~\ref{sec:pre_meth}. 


\begin{table}[t!]
  \small
  \centering
   \setlength{\tabcolsep}{9pt}
  \caption{\label{tab:maintainers}Performance of \slic. (Validation with Owners\Space{ of OSS projects})}
  \vspace{-2ex}
  \begin{tabular}{lrrr} 
    \toprule
    Rule & \#TP & \#FP  & Precision\\
    \midrule
    Hard-coded secrets & $77$ & $119$ & $0.39$ \\
    Use of HTTP without TLS & $1$ & $72$ & $0.01$\\
    Suspicious comments & $3$ & $15$ & $0.17$\\
    Use of Weak Crypto. Algos. & $0$ & $3$ & $0.00$\\
    Invalid IP Address Binding & $0$ & $1$ & $0.00$\\
    Empty Password & $1$ & $5$ & $0.17$ \\
    Admin by default & $1$ & $0$ & $1.00$\\
    \midrule
    Total & 83 & 215 & 0.28\\
    \bottomrule 
  \end{tabular}
  \vspace{-2ex}  
\end{table}

\subsubsection{Results} This section reports results. \textbf{Issues.} We
reported a total of \botTotalWarnings{} warnings in \botTotalIssues{} issues
($9$ warnings per issue, on average). Project owners responded 
to \botTotalIssuesAnswers{} issues of the \botTotalIssues{}
issues we submitted, but only \botFinalIssues{} issues were clearly discussed 
or validated---the equivalent to \botTotalAnswered{} warnings (Table~\ref{tab:maintainers}). One issue for an
``Empty Password'' warning was fixed by one of the maintainers
(82c3cb7\footnote{Fix for ``Empty password'' issue: \url{https://github.com/jtopjian/puppet-sshkeys/commit/82c3cb7e78c16cf6517207779f79ab5b2a71b603} (Accessed \today)});
one tagged the issue with ``waiting for contribution''; another commented asking to perform a pull request. \textbf{Reply
  Evaluation.} We used the following method to determine the warning
classification (\ie{}, false or true positive) from the answers of
project owners. We discarded warnings related to issues closed without
any response or activity and issues that remained open or without any
response by the time of our analysis. After that stage, two of this paper's co-authors reviewed each of the answered
issues. Each warning received two votes. Then, we ran a Cohen's Kappa
analysis to measure the inter-rater reliability of our choices to
assess the confidence in our classification method. The kappa
coefficient ($k$) shows the level of agreement between the two
co-authors. The analysis yielded $k=1.0$, i.e., a total agreement 
between both co-authors. \textbf{Precision.} Table~\ref{tab:maintainers} shows the
number of true positives~(TP) and the number of false positives~(FP) per
type of warning. In total, \slic\ reported $83$ true positives and
$215$ false positives for the $33$ issues considered, which resulted in
an average precision of $0.28$ for \slic{}. Note that the samples used for 
``Use of Weak Crypto. Algorithms'', ``Invalid IP Address Binding'', ``Empty Password'' and 
``Admin by default'' are relatively small, so results might not reflect the entire reality. 


\textit{\textbf{Summary:}} Results indicate that the precision of
\slic{} is even lower when evaluated by maintainers---developers with 
more knowledge and context of the applications---of the software (drops to $28\%$).
These results confirm our initial observations and indicate that 
better security IaC linters for Puppet are needed.

\section{\toolname: Puppet Security Linter}\label{sec:tool}

The observations described in the previous section motivated the 
pursuit of a more precise security linter for Puppet scripts. 
The previous experiments ignited discussions with members 
of the development and security teams of PuppetLabs, as well 
as a project manager from Vox Pupuli. We leveraged the feedback 
obtained from the previous studies and the professional feedback to
design a new security linter for the Puppet
community, which we dubbed as \toolname{}. More precisely, we created 
a new linter as a result of \textit{phase 1} (Figure~\ref{fig:timeline}) 
and incrementally evolved the linter according to the recommendations of professionals 
(Figure~\ref{fig:timeline}, \textit{phase 2}), improving
\noRulesSlic\ rules of the \slic{} ruleset and adding
\newRules\ new rules. 
The following sections report the new architecture (Section~\ref{sec:architecture}), 
design choices (Section~\ref{sec:design}) and security checkers (Section~\ref{sec:rules}) that resulted
from the feedback collection (Figure~\ref{fig:timeline}): 
\begin{itemize}
  \item \textit{Phase 1}: feedback from the \textbf{owners of OSS projects},  
  \textbf{PuppetLabs} and \textbf{Voxpupuli Engineers} (as described in 
  Section~\ref{sec:preliminary_study}) that led to the creation of \toolname{} (v0.1.0);
  \item \textit{Phase 2}: two cycles of feedback from the \textbf{Puppet} and \textbf{Prolific} 
  communities (as described in Section~\ref{sec:evaluation}) that led to two new releases of 
  \toolname{} (v1.0.0 and v1.1.0);
\end{itemize}


\subsection{Linter Architecture}\label{sec:architecture}

One recommendation from the PuppetLabs team (\textit{phase 1}) was to implement 
the set of the rules as plugins to the \puplint{} architecture\footnote{Puppet-lint website: \url{http://puppet-lint.com/}},
through the \puplint\ check API\footnote{Puppet-lint check API: \url{http://puppet-lint.com/developer/api/}}. 
This API facilitates the integration of new checkers in \puplint\. 
In addition, it allows the user to suppress warnings and disable or enable
checkers---which are regarded as important features by the community. All 
security checks were developed as plugins to \puplint\ (Table~\ref{tab:rules}). 
These checks are applied to the Abstract Syntax Tree (AST) of a Puppet manifest which 
is generated by an internal tokenizer\footnote{Puppet-lint tokenizer: \url{http://puppet-lint.com/developer/tokens/}}.
\toolname\ was implemented in \texttt{Ruby} and its \texttt{CLI} is available 
online\footnote{Gem is available at \url{https://rubygems.org/gems/puppet-lint-infrasecure}}.
The codebase of the linter is available 
at \url{https://github.com/TQRG/puppet-lint-infrasecure} and open to future contributions.

\subsection{Design Choices}\label{sec:design}

This section describes the design choices of our analysis, guided by 
the distinct cycles of feedback as described in Section~\ref{sec:preliminary_study} 
and Section~\ref{sec:evaluation}; also, illustrated in Figure~\ref{fig:timeline}.

\textbf{Variable/Attribute Assignments (VASS).} From the preliminary 
analysis performed in Section~\ref{sec:preliminary_study}, we have noticed 
security-related code smells being detected in logical conditions. For instance, 
\CodeIn{if has\_key(\$userdata, ’env’)} shows a logical condition that 
was incorrectly flagged as a hard-coded secret issue. Aiming to 
reduce the number of incorrect predictions, we implemented a rule to search 
for variable and attribute assignments in Puppet manifests---\texttt{isVarAssign(\textit{token})}
and \texttt{isAtrAssign(\textit{token})} (cf. Table~\ref{tab:rules}).

\textbf{Reasoning about the token value (TOKVAL).} Some of the rules did not 
reason about \texttt{token.value}. For hard-coded secrets, 
the linter only checks if the token value is not empty. While manually validating the 
samples used in our studies, we found false positives of hard-coded secrets. For instance, 
\CodeIn{aws\_admin\_username = downcase(\$::operatingsystem)} which does not store 
any actual secret. \slic{} flagged this case as a hard-coded secret because the value assigned 
to the variable \CodeIn{aws\_admin\_username} is not empty. However, the rule needs to 
reason not only about the length of the right-hand side of the variable assignment but also about 
the type of token and value. \toolname\ locates variable and attribute assignments in the AST and considers 
that secrets are usually stored in \CodeIn{:STRING} and \CodeIn{:SSTRING} tokens. In addition, we defined a database
of known credentials (\texttt{isUserDefault(\textit{token.value})})---credentials that are not 
considered secrets by the community\footnote{\url{https://puppet.com/docs/pe/2019.8/what\_gets\_installed\_and\_where.html\#user\_and\_group\_accounts\_installed}}---and,
invalid secrets (\texttt{invalidSecret(\textit{token.value})}) which are 
consider as non-valid values for hard-coded secrets. The linter ignores all the credentials in this database. Feedback from distinct \textbf{owners of OSS Projects} is what
drove us to make this decision is presented below:

\begin{itemize}[topsep=.2ex,itemsep=.2ex,leftmargin=0em]
  \item[] \textbf{[User Default]}: 
  \textit{``The names of these UNIX accounts are not 
  considered to be secret. They are 
  published openly as part of the PE documentation:
  \url{https://puppet.com/docs/pe/2019.8/what\_gets\_installed\_and\_where.html\#user\_and\_group\_accounts\_installed}''}
  \item[] \textbf{[Invalid Secret]}: 
  \textit{``This are default users and default as found in every installed 
  fpm package. there is most of the time a wwwrun or a www-data user 
  depending on the system.''}
\end{itemize}

\textbf{Use of HTTP without TLS is fine sometimes (SAFE).} As \slic{}, \toolname{} also flags 
every single occurrence of \textit{http://}, i.e., it recommends to use TLS by default. For example, the tool
flags \CodeIn{apturl => "http://deb.debian.org/debian"}, even though it refers to a credible 
source. Our definition of credible source is a source that can be trusted. However, 
different companies can have different opinions regarding the credibility of the same source.
That is why this rule is customizable. We observed that this type of issues (CWE-319) are prevalent in Puppet files. 
Applications often use third-party libraries, which are usually configured in Puppet 
files, and the links to their sources are not necessarily unsafe.
Also, depending on the context of an application, the configuration of 
localhost servers as HTTP may not be a problem. If no sensitive data is communicated, 
then there is probably no problem using \CodeIn{http}. \toolname{} has a configuration file 
for safe domains, i.e., domains that are cleared to be use \CodeIn{http}. Thereby, infrastructure teams can 
customize their own configuration files. The feedback provided 
in Section~\ref{sec:evaluation} from two different \textbf{practitioners}, which
supported this decision, is presented bellow:

\begin{itemize}[topsep=.2ex,itemsep=.2ex,leftmargin=0em]
  \item[] \textbf{[Whitelist]}: 
  \textit{``I think it is fine if localhost is used. Otherwise TLS 
  should be mandatory. All the big 
  financial organizations will not use this check because 
  they cannot create internal certs or use 
  letsencrypt.''}
  \item[] \textbf{[Whitelist]}: 
  \textit{``By default, it's unsafe to not use HTTPS. 
  But for internal testing/development it is acceptable 
  to me to not use HTTPS all the time.''}
\end{itemize}


\textbf{Hard-Coded Secret Division in different checkers (SECR).} 
In Section~\ref{sec:preliminary_study},
we observed that the hard-coded secrets checker produces the most significant number
of alerts. For instance, \slic{} assumes a secret is a key, password or username. 
As mentioned previously in Section~\ref{sec:background},
the Common Weakness Enumeration list does not consider solo 
hard-coded usernames as a threat. Practitioners involved in our validation 
studies shared the same opinion. 
We analyzed the distribution of the different types of hard-coded secrets
and realized that $48\%$ of the secrets detected were usernames.
Therefore, in the final version of our tool, we decided
to separate the hard-coded secrets checker into three new checkers (one per type of secret). This 
way, developers can disable the username checker if they find it 
noisy. We did not delete the original checker; infrastructure
teams can use it if they want to collect all the different 
types of secrets simultaneously. Feedback provided 
in Section~\ref{sec:evaluation} from a \textbf{practitioner} 
supported this decision:

\begin{itemize}[topsep=.2ex,itemsep=.2ex,leftmargin=0em]
  \item[] \textbf{[Username]}: 
  \textit{``The main security issue is having the password hard-coded. 
  About having the user hard-coded, it is possible to allow that as an 
  initial setting that should be changed during the first configuration and, 
  in that case, it is not so much a security issue.''}
\end{itemize}


\begin{table}
  \centering
  \small
  \caption{\toolname{}'s list of string and AST patterns.}\label{tab:pattterns}
  \vspace{-2ex}
  \renewcommand{\arraystretch}{0.5}
  \begin{subtable}[h]{\linewidth}
    \begin{tabular}{p{3cm}p{5cm}} 
      \toprule
      \textbf{Rule} & \textbf{String Pattern}  \\
      \toprule
          isAdmin(\textit{t.value}) &  
            \url{root|admin} \\ \midrule
          isNonSecret(\textit{t.value}) & 
            \url{gpg|path|type|buff|zone|mode|tag|header|scheme|length|guid} \\\midrule
          isPassword(\textit{t.value}) & 
            \url{pass(word|\_|$)|pwd} \\\midrule
          isUser(\textit{t.value}) & 
            \url{user|usr} \\\midrule
          isKey(\textit{t.value}) & 
            \url{(pvt|priv)+.*(cert|key|rsa|secret|ssl)+} \\\midrule
          isPlaceholder(\textit{t.value}) & 
            \url{${.*}|($)?.*::.*(::)?} \\\midrule
          hasCyrillic(\textit{t.value}) & 
            \url{^(http(s)?://)?.*\\p{Cyrillic}+}\\ \midrule
          isInvalidIPBind(\textit{t.value}) & 
            \url{^((http(s)?://)?0.0.0.0(:\\d{1,5})?)$} \\ \midrule
          isSuspiciousWord(\textit{t.value}) & 
            \url{hack|fixme|ticket|bug|checkme|secur|debug|defect|weak} \\ \midrule
          isWeakCrypto(\textit{t.value}) & 
            \url{^(sha1|md5)} \\ \midrule
          isCheckSum(\textit{t.value}) & 
            \url{checksum|gpg} \\ \midrule
          isHTTP(\textit{t.value}) & 
            \url{^http://.+} \\ \midrule
          isUserDefault(\textit{t.value}) & 
            \url{pe-puppet|pe-webserver|pe-puppetdb|pe-postgres|pe-console-services|pe-orchestration-services|pe-ace-server|pe-bolt-server} \\ \midrule
          invalidSecret(\textit{t.value}) & 
            \url{undefined|unset|www-data|wwwrun|www|no|yes|[]|undef|true|false|changeit|changeme|none} \\\midrule
          isStrongPwd(\textit{t.value})~\footnote{The \texttt{strong\_password} ruby gem (\url{https://rubygems.org/gems/strong_password}) is used to determine if a password is strong or not.} & 
            StrongPassword::StrengthChecker(\textit{t.value}) \\\midrule
          isEmptyPassword(\textit{t.value}) & 
            t.value == ``''\\\midrule
          isVersion(\textit{t.value}) & 
            \url{.*_version}\\
        \bottomrule
      \end{tabular}
      \caption{String patterns are applied to token values.}
      \label{tab:string_patterns}
      \end{subtable}

      \par\bigskip

      \begin{subtable}[h]{\linewidth}
      \centering
      \begin{tabular}{p{2cm}p{6cm}} 
        \toprule
        \textbf{Rule} & \textbf{AST Pattern}  \\
        \midrule		
        isVariable(\textit{t}) & 
          t.type == \texttt{:VARIABLE} $\vee$ t.type == \texttt{:NAME} \\\midrule
        isString(\textit{t}) & 
          t.type == \texttt{:STRING} $\vee$ t.type == \texttt{:SSTRING} \\\midrule
        isVarAssign(\textit{t}) & 
          isVariable(\textit{t.prev\_code\_token}) $\wedge$ 
          t.type == \texttt{:EQUALS} $\wedge$ 
          isString(\textit{t.next\_code\_token}) \\\midrule
        isAtrAssign(\textit{t}) & 
          isVariable(\textit{t.prev\_code\_token}) $\wedge$ 
          t.type == \texttt{:FARROW} $\wedge$ 
          isString(\textit{t.next\_code\_token}) \\\midrule
        isResource(\textit{t}) &
          (t.prev\_code\_t.type == \texttt{:NAME} $\wedge$ 
          t.type == \texttt{:LBRACE} $\wedge$ 
          t.next\_code\_t.type == \texttt{:SSTRING}) $\vee$
          (t.prev\_code\_t.type == \texttt{:LBRACE} $\wedge$ 
          t.type == \texttt{:SSTRING})\\\midrule
        isFunctionCall(\textit{t}) &
          (t.type == \texttt{:NAME} $\wedge$ 
          t.next\_code\_token.type == \texttt{:LPAREN}) $\vee$
          t.type == \texttt{:FUNCTION\_NAME} \\\midrule
        isComment(\textit{t}) & 
          t.type is in (\texttt{:COMMENT}, \texttt{:MLCOMMENT}, \texttt{:SLASH\_COMMENT})\\
      \bottomrule
      \end{tabular}
      \caption{Patterns applied to the Abstract Syntax Tree (AST).}
      \label{tab:ast_patterns}
      \end{subtable}
      \vspace{-4ex}
\end{table}
%
%
\subsection{Rules}\label{sec:rules}
%
\toolname{} detects $12$ different security smells in Puppet manifests.
Table~\ref{tab:ast_patterns} presents the AST patterns 
that are searched in the AST for relevant nodes/sequences of nodes; 
and, table~\ref{tab:string_patterns} presents the string patterns used 
to validate the information in those nodes.
Table~\ref{tab:rules} shows the syntactic pattern matching rules per 
weakness which leverage the two sets of patterns mentioned
before.

\textbf{Hard-coded secrets (CWE-321, CWE-259, CWE-798):} The top of the Table~\ref{tab:rules}
contains $4$ different rules for hard-coded secrets: one per secret;
and a final one which detects all kinds of secrets at the same time (keys, 
password and usernames). In addition to the design choices,  
the rules consider that secret values cannot be placeholders 
(\CodeIn{!isPlaceholder()}, Table~\ref{tab:string_patterns}).

\textbf{Use of HTTP without TLS (CWE-319):} One of the main findings 
of our analysis is that HTTP without TLS is not always problematic. Therefore,
we created a configurable whitelist where infrastructure teams can add
safe domains. The checker will not raise alerts when in the 
presence of a safe domain (\CodeIn{inWhitelist()}, Table~\ref{tab:rules}). \toolname{} provides a default whitelist with 
known reliable sources such as \url{http://deb.debian.org/debian}. 
However, this default whitelist will be overwritten if the user 
configures a new one.


\begin{table}[t!]
  \small
  \centering
  \setlength{\tabcolsep}{4pt}
  \caption{Performance of \toolname\ v0.1.0.}
  \label{tab:prel_analysis_infrasecure}
  \vspace{-2ex}  
  \begin{tabular}{lrrrrrr} 
    \toprule
    \cellcolor{Gray} \textbf{\toolname{} v0.1.0} & \multicolumn{3}{c}{\textit{proportional}} & \multicolumn{3}{c}{\textit{uniform}}\\\midrule
    Rule & \#TP & \#FP & Pr. &  \#TP & \#FP & Pr. \\
    \midrule
    Hard-coded secrets & \tpHardcodedSecretsInfraSecureProportional{} & \fpHardcodedSecretsInfraSecureProportional{} & \precHardcodedSecretsInfraSecureProportional{} & 
    \tpHardcodedSecretsInfraSecureUniform{} & \fpHardcodedSecretsInfraSecureUniform{} & \precHardcodedSecretsInfraSecureUniform{} \\
    Use of HTTP without TLS & \tpHttpWithoutTLSInfraSecureProportional{} & \fpHttpWithoutTLSInfraSecureProportional{} & \precHttpWithoutTLSInfraSecureProportional{} & 
    \tpHttpWithoutTLSInfraSecureUniform{} & \fpHttpWithoutTLSInfraSecureUniform{} & \precHttpWithoutTLSInfraSecureUniform{} \\
    Suspicious comments & \tpSuspiciousCommentsInfraSecureProportional{} & \fpSuspiciousCommentsInfraSecureProportional{} & \precSuspiciousCommentsInfraSecureProportional{} & 
    \tpSuspiciousCommentsInfraSecureUniform{} & \fpSuspiciousCommentsInfraSecureUniform{} & \precSuspiciousCommentsInfraSecureUniform{} \\
    Use of Weak Crypto. Algorithms & \tpWeakCryptoInfraSecureProportional{} & \fpWeakCryptoInfraSecureProportional{} & \precWeakCryptoInfraSecureProportional{} & 
    \tpWeakCryptoInfraSecureUniform{} & \fpWeakCryptoInfraSecureUniform{} & \precWeakCryptoInfraSecureUniform{} \\
    Invalid IP Address Binding & \tpInvalidIPInfraSecureProportional{} & \fpInvalidIPInfraSecureProportional{} & \precInvalidIPInfraSecureProportional{} &
    \tpInvalidIPInfraSecureUniform{} & \fpInvalidIPInfraSecureUniform{} & \precInvalidIPInfraSecureUniform{} \\
    Empty Password & \tpEmptyPassInfraSecureProportional{} & \fpEmptyPassInfraSecureProportional{} & \precEmptyPassInfraSecureProportional{} & 
    \tpEmptyPassInfraSecureUniform{} & \fpEmptyPassInfraSecureUniform{} & \precEmptyPassInfraSecureUniform{} \\
    Admin by default & \tpAdminDefaultInfraSecureProportional{} & \fpAdminDefaultInfraSecureProportional{} & \precAdminDefaultInfraSecureProportional{} & 
    \tpAdminDefaultInfraSecureUniform{} & \fpAdminDefaultInfraSecureUniform{} & \precAdminDefaultInfraSecureUniform{}\\
    \midrule
    Total & \tpInfraSecureProportionalSample{} & \fpInfraSecureProportionalSample{} & \precTotalInfraSecureProportional{} & \tpInfraSecureUniformSample{} & \fpInfraSecureUniformSample{} & \precTotalInfraSecureUniform{} \\
    \bottomrule 
  \end{tabular}
  \vspace{-3ex}  
\end{table}


\textbf{Suspicious Comments (CWE-546):} This checker was controversial. It was 
recognized that it would be valuable to alert developers about comments in their code mentioning 
functionalities and weaknesses that might hint at attackers. However, keywords such 
as ``todo'', ``later'', and ``later2'' were considered noisy.  
We changed the list of keywords in response to the complaints and feedback 
obtained from the developers (\CodeIn{isSuspiciousWord()}, Table~\ref{tab:string_patterns}).

\textbf{Usage of Weak Crypto. Algorithms (CWE-326):} \toolname{}
searches for \emph{in calls to} functions (\CodeIn{isFunctionCall(), Table~\ref{tab:ast_patterns}}) implementing 
crypto algorithms such as ``md5'' and ``sha1''
in variable and attribute assignments (Table~\ref{tab:rules}). 

\textbf{Invalid IP Address Binding (CWE-284):} We found cases where 
the invalid IP \texttt{0.0.0.0} was in descriptions and commands. 
For instance, \slic{} flags \CodeIn{description => 'Open up postgresql for access to 
sensu from 0.0.0.0/0'}. IPs follow a 
dot-decimal notation, i.e., they should not include letters. 
\toolname\ uses a less naive regex 
than the string pattern 
(\CodeIn{isInvalidIPBind()}, Table~\ref{tab:string_patterns}).

\textbf{Empty Password (CWE-258):} Empty passwords 
are located the same way as hard-coded secrets, i.e., 
by focusing on variable and attribute assignments (Table~\ref{tab:rules}).
The rule \CodeIn{isEmptyPassword()} (Table~\ref{tab:string_patterns}) verifies 
if the password is empty. 

\textbf{Admin By Default (CWE-250):} These issues 
are also located by focusing on variable and attribute 
assignments (Table~\ref{tab:rules}). The rule \CodeIn{isAdmin()}, table~\ref{tab:string_patterns}, verifies 
if the user is ``admin'' or ``root''.


\textbf{Homograph Attacks (CWE-1007):} Typosquatting
attacks, also known as URL hijacking, is a social engineering attack 
that purposely uses misspelt domains for malicious purposes;
and are the cause of many supply chain attacks~\cite{duan2020measuring}.
This checker is important because malicious actors 
can use homoglyphs to modify reliable sources 
for malicious sources (\CodeIn{hasCyrillic()}, Table~\ref{tab:string_patterns}).


\textbf{Weak Password (CWE-521):} \toolname{} searches 
for passwords in the same way it searches for Empty Passwords and Hard-Coded 
Passwords. The only difference is the password value validation (\CodeIn{isStrongPwd()}, Table~\ref{tab:string_patterns})
which is performed by an external package (\texttt{strong\_password})
that implements an adaptation of a PHP algorithm developed by Thomas Hruska~\cite{blogpost}.

\textbf{Malicious Dependencies (CWE-829):} We produced a database of 
malicious dependencies for 
Puppet modules by crossing CVEs information and 
vulnerable products names with third-party libraries 
that can be configured in Puppet manifests.
We used the National Vulnerability Database (NVD)
to collect the CVEs and respective vulnerable 
products---from the list of Known Affected Software Configurations.
To get the list of products used by Puppet,
we used the Forge API\footnote{Forge API is available 
at https://forgeapi.puppet.com/}. 
Our database integrates malicious dependencies
for $33$ different packages (e.g., \texttt{rabbitmq}, 
\texttt{apt}, \texttt{cassandra}, \texttt{postgresql}, etc).
The checker searches for resource configurations (\CodeIn{isResource()}, Table~\ref{tab:ast_patterns})
and verifies if the 
a configured version of the software integrates 
our database of malicious dependencies for Puppet (\CodeIn{isMalicious()}, Table~\ref{tab:rules}).



\begin{table*}[h]
  \small
  \centering
  \caption{\toolname\ rules to detect security smells.}
  \vspace{-2ex}
  \setlength{\tabcolsep}{3pt} % General space between cols (6pt standard)
  \renewcommand{\arraystretch}{0.9} % General space between rows (1 standard)
      \begin{tabular}{p{1.3cm}p{3cm}p{12.5cm}} 
    \toprule
  \textbf{CWE} & \textbf{Weakness Name}	& \textbf{Rule} \\
  \midrule		
  CWE-321 & Hard-coded Key & 
    (isVarAssign(\textit{t}) $\vee$ isAtrAssign(\textit{t})) $\wedge$
    isKey(\textit{t.prev\_code\_token}) $\wedge$ 
    isNonSecret(\textit{t.prev\_code\_token}) $\wedge$ 
    !isPlaceholder(\textit{t.next\_code\_token}) \\\midrule
  CWE-259 & Hard-coded Password & 
    (isVarAssign(\textit{t}) $\vee$ isAtrAssign(\textit{t})) $\wedge$
    isPassword(\textit{t.prev\_code\_token}) $\wedge$ 
    isNonSecret(\textit{t.prev\_code\_token}) $\wedge$ 
    !isPlaceholder(\textit{t.next\_code\_token}) $\wedge$ 
    !isUserDefault(\textit{t.next\_code\_token}) $\wedge$
    !invalidSecret(\textit{t.next\_code\_token})\\\midrule
  CWE-798 & Hard-coded Usernames & 
    (isVarAssign(\textit{t}) $\vee$ isAtrAssign(\textit{t})) $\wedge$
    isUser(\textit{t.prev\_code\_token}) $\wedge$ 
    isNonSecret(\textit{t.prev\_code\_token}) $\wedge$ 
    !isPlaceholder(\textit{t.next\_code\_token}) $\wedge$ 
    !isUserDefault(\textit{t.next\_code\_token}) $\wedge$
    !invalidSecret(\textit{t.next\_code\_token}) \\\midrule
  CWE-798 & Hard-coded Secrets & 
    (isVarAssign(\textit{t}) $\vee$ isAtrAssign(\textit{t})) $\wedge$
    (isKey(\textit{t.prev\_code\_token}) $\vee$ 
    isPassword(\textit{t.prev\_code\_token}) $\vee$
    isUser(\textit{t.prev\_code\_token})) $\wedge$
    !isPlaceholder(\textit{t.next\_code\_token}) $\wedge$ 
    !isUserDefault(\textit{t.next\_code\_token}) $\wedge$
    !invalidSecret(\textit{t.next\_code\_token}) \\\midrule  
  CWE-319 & Use of HTTP without TLS & 
    (isVarAssign(\textit{t}) $\vee$ isAtrAssign(\textit{t})) $\wedge$
    isHTTP(\textit{t.next\_code\_token}) $\wedge$
    !inWhitelist(\textit{t.next\_code\_token})\\\midrule
  CWE-546 & Suspicious Comments & 
    isComment(\textit{t}) $\wedge$ 
    isSuspiciousWord(\textit{t})\\\midrule
  CWE-326 & Use of Weak Crypto. Algo. & 
    (isVarAssign(\textit{t.prev\_code\_token}) $\vee$ 
    isAtrAssign(\textit{t.prev\_code\_token}) $\vee$ 
    isFunctionCall(\textit{t.next\_code\_token})) $\wedge$
    !isCheckSum(\textit{t.prev\_code\_token}) $\wedge$
    isWeakCrypto(\textit{t.next\_code\_token}) \\\midrule
  CWE-284 & Invalid IP Address Binding & 
    (isVarAssign(\textit{t}) $\vee$ isAtrAssign(\textit{t})) $\wedge$
    isInvalidIPBind(\textit{t.next\_code\_token})\\\midrule
  CWE-258 & Empty Password & 
    (isVarAssign(\textit{t}) $\vee$ isAtrAssign(\textit{t})) $\wedge$ 
    isPassword(\textit{t.prev\_code\_token}) $\wedge$ 
    isEmptyPassword(t.prev\_code\_token)\\\midrule
  CWE-250 & Admin by default & 
    (isVarAssign(\textit{t}) $\vee$ isAtrAssign(\textit{t})) $\wedge$ 
    isNonSecret(\textit{t.prev\_code\_token}) $\wedge$ 
    isUser(\textit{t.prev\_code\_token}) $\wedge$ 
    !isPlaceholder(\textit{t.next\_code\_token}) $\wedge$ 
    isAdmin(\textit{t.next\_code\_token})\\\midrule
  CWE-1007 & Homograph Attacks & 
    (isVarAssign(\textit{t}) $\vee$ isAtrAssign(\textit{t})) $\wedge$ 
    hasCyrillic(\textit{t.next\_code\_token})\\\midrule
  CWE-521 & Weak Password & 
    (isVarAssign(\textit{t}) $\vee$ 
    isAtrAssign(\textit{t})) $\wedge$ 
    isPassword(\textit{t.prev\_code\_token}) $\wedge$ 
    isStrongPwd(\textit{t.next\_code\_token})\\\midrule
  CWE-829 & Malicious Dependencies & 
    isResource(\textit{t}) $\wedge$ 
    isVersion(\textit{t.prev\_code\_token}) $\wedge$
    isMalicious(\textit{t.next\_code\_token})\\
  \bottomrule
  \multicolumn{3}{l}{\setlength{\tabcolsep}{12pt}\CodeIn{inWhitelist(t.value)} verifies if the URL is in 
  the list of configurable safe domains/whitelist. If the URL 
  is in the whitelist, an alert should not be raised.}  \\  
  \multicolumn{3}{l}{\setlength{\tabcolsep}{12pt}\CodeIn{isMalicious(t.value)} verifies if the software package 
  version configured in the puppet script is in the database of malicious dependencies.}  \\  
  \bottomrule
  \end{tabular}
  \label{tab:rules}
  \vspace{-2ex}
\end{table*}

\subsection{Proof of Concept: \toolname{} v0.1.0}
%
As a proof of concept, two of the design choices described in Section~\ref{sec:design}
were implemented in the first version, \toolname{} v0.1.0, to 
ascertain whether precision could be enhanced. In particular, we focused on implementing 
variable and attribute assignments (VASS) and reasoning about the token value (TOKVAL),  
to reduce the number of incorrect detections. 

In our preliminary analysis with students (see Section~\ref{sec:preliminary_study}, 
Table~\ref{tab:prel_analysis_slic}), we observed that the precision of \slic{} was 
$64\%$. By implementing the two design choices mentioned before, we
increased precision by $12$ per cent points---when comparing \slic{}'s precision in
the \textit{proportional} set ($64\%$) with the precision of the first version of 
\toolname{} in the same dataset ($76\%$), Table~\ref{tab:prel_analysis_infrasecure}.
As these changes were successful w.r.t. precision, we decided to implement 
the other improvements and conduct a new study with practitioners to collect 
more feedback about the tool and the anti-patterns covered.


\section{Practitioners Evaluation}\label{sec:evaluation}

\toolname\ was validated with practitioners experienced in security 
or configuration management technologies. We built an experiment to validate 
the warnings of the new tool. The experiment was shared with the Puppet 
communities on Slack
(\texttt{puppetcommunity.slack.com}) and Reddit (\texttt{r/puppet}).
We found \noProfessionalsCommunity\ participants by this mean. Later, 
we leveraged Prolific\footnote{Prolific Platform: https://www.prolific.co/}~\cite{arxiv.2201.05348} 
to gather more participants
based on their experience and programming knowledge. 
In this experiment, a total of \totalWarningsPrac\ warnings were validated 
by \noProfessionals~practitioners. Furthermore, our improvements increased 
the precision of the tool from \botPrecision\ to \finalPrecision.
As illustrated in Figure~\ref{fig:timeline}, we run
two cycles of feedback collection and iteratively 
improve the tool with the feedback collected.
%
This section describes 1) the methodology 
conducted with practitioners to validate the \toolname\ 
warnings; and 2) the results obtained 
from running the practitioners' experiment.

\subsection{Study Design}

In this section, we detail how the validation study of \toolname\ 
was designed, and the population leveraged to conduct it.
\toolname\ was improved based on the problems collected through 
the preliminary study and the validations with the maintainers 
of the software---which led to \toolname{} v0.1.0. To validate the new tool, 
we surveyed practitioners with experience in security, configuration management 
tools and programming knowledge by following 
recent recommendations to run studies on Prolific~\cite{arxiv.2201.05348}. 
After the pre-screening, the practitioners 
were asked to validate and give feedback on \noWarningsPerPracticioner\ different 
warnings generated by \toolname.

\textbf{Practitioners Recruitment.} The participants were obtained using
two distinct routes: 1) By sharing the study with online Puppet communities such as 
    \texttt{puppetcommunity.slack.com} (slack) and \texttt{r/puppet} (reddit); 2) By using the Prolific platform to gather practitioners with 
    experience in security, configuration management tools and programming skills.
Both communities integrate a considerable amount of members: slack has 
over $9$k members, and Reddit has around $4.7$k members. However, only \noProfessionalsCommunity\ 
members participated in our study.
Therefore, we used Prolific to collect more practitioners with experience in security 
and configuration management tools outside of these two communities. 
Prolific participants were monetarily compensated for answering each survey, while the 
participants collected in the communities were not.

\textbf{Pre-Screening.} Prolific 
is a platform where you can find participants to perform online research. As recommended 
in research on recruiting practitioners for user studies on prolific~\cite{arxiv.2201.05348}, we performed 
a pre-screening of the population to collect adequate participants for this study, 
i.e., participants with security and configuration management experience; and 
programming knowledge. Prolific has filters dedicated to the industry where the 
participants work or belong. We sent the pre-screening survey to prolific 
users working in the following industries: ``Computer and Electronics Manufacturing'',
``Information Services and Data Processing'', ``Product Development'',
``Research laboratories'', ``Scientific or Technical Services'',
``Software''. The participants were asked to answer the following questions:
1) Do you have any kind of experience with configuration management tools?
\textbf{Choices:} Puppet, Ansible, Terraform, Chef, Other;
2) Experience in Security (Number of Years); 
3) Experience in Infrastructure as a Service (Number of Years);
and three programming language questions about different puppet
configurations. Due to space constraints, we do not present the
questions here, but they are available in our replication 
package: \url{study/practitioners/pre-screening/puppet-study-form.pdf}. 

We obtained a total of \noProfessionalsProlificRaw\ responses 
from $8$ different industries. Then, we ordered those participants by priority
where priority is the count of experience in 1) at least one configuration management tool ($CMEXP$), 
2) security ($SECEXP$), 3) infrastructure as a service ($INFRAEXP$); and 4) score in the programming 
questions ($SCORE$). 
Priority was calculated as follows $0.3 * ((CMEXP + SECEXP + INFRAEXP)/3) + 0.7 * (SCORE/3)$
and varies between $0$ to $3$. A priority of 
$3$ means the participant is adequate for the study, whereas a priority of $0$ means the participant 
is not adequate. For the validation study, we only invited  
participants with priority equal to or greater than $1.5$---which represented $53\%$ of the
initial responses (227 out of 431 participants).


\textbf{Validation Form.} We built a form online to share with the Puppet communities 
and practitioners. The initial page of the form explains the study's goal and asks 
the participant for her profession/career, number of years of experience in security, and number of years of experience in infrastructure/puppet. 
The goal of the study is to validate the output of our new tool: \toolname. 
Therefore, participants are required to validate $3$ different warnings (one at each time). For each warning, 
the form presents a description of the issue and the piece of code where the issue is located (cf. Figure~\ref{fig:warning}). 
Participants have to evaluate the issue and provide their validation:
``Yes, I agree'', if the warning reports a security issue; ``No, I disagree'',
if it reports a false security issue; or, ``I'm not sure'', when unsure.
The participant can also provide additional feedback on the problem.

\textbf{Warnings Dataset.}
For this experiment, we validated the output of $9$ different rules (Table~\ref{tab:rules}), where 
the warnings for weak passwords and malicious dependencies were mostly validated in 
the second round of feedback collection. We ran the \toolname\
over a total of $1050$ GitHub projects---collected from the dataset 
used in the preliminary study (Section~\ref{sec:preliminary_study}). We created
a uniform sample with $50$ warnings per rule (i.e., a total of $450$ warnings).

\begin{figure}[t!]
    \centering
    \includegraphics[width=1.05\linewidth]{warning.png}
    \vspace{-3ex}
    \caption{Example of the form presented to the 
    practicioner for warning validation.}\label{fig:warning}
    \vspace{-3ex}
\end{figure}

\textbf{Metrics.} We report the number of true positives (TPs), 
the number of false positives (TPs), the number of ``Unsure'' 
responses and Precision---calculated as described in 
Section~\ref{sec:pre_meth}.

\subsection{Results}
We obtained a total of \noProfessionals{} participants: $74$ in 
the first round of feedback; and $57$ in the second 
round. Due to the lack of responses from participants, we 
were only able to validate $342$ out of the 
$450$ initial number of warnings.
Table~\ref{tab:final_prac} shows the distribution of warnings and 
precision obtained for the final version of \toolname{} (v1.1.0). 
At the end of this study, \toolname{} reported a precision of 
\finalPrecision{}, where $54$ of the warnings  
were False Positives. 

Part of the feedback obtained in this experiment
was documented in Section~\ref{sec:design} and~\ref{sec:rules}.
Table~\ref{tab:infrasecure_prec} shows the evolution 
of the tool's precision with the different iterations of feedback.
In contrast to Table~\ref{tab:prel_analysis_infrasecure}, where 
we report the precision of v0.1.0 for the alerts validated by students;
here, in Table~\ref{tab:infrasecure_prec}, we report the precision of v0.1.0
based on the practitioners' feedback, i.e., leveraging the alerts validated 
by practitioners (instead of students). It is important to 
note that the implementation of v0.1.0 focused on understanding 
variable and attribute assignments and reasoning about the token value  
to reduce the number of incorrect detections. These two improvements affected 
all checkers. The remaining versions of the tool focused on addressing specific 
false positives, extending the ruleset and adding the safe domains feature. 
In the comparison provided in Table~\ref{tab:infrasecure_prec}, we 
observe an increase of precision---from $76\%$
to $83\%$---by conducting different cycles of 
feedback collection.
In addition, feedback was essential to extend 
the ruleset. This study with practitioners led us 
to create $3$ new rules to detect weak passwords;
typosquatting attacks; and malicious dependencies (being 
the last two the root causes of many supply chain attacks~\cite{9402108,duan2020measuring}).

\textit{\textbf{Summary:}} Results show that working side-by-side with the community
will help the authors of the tools develop better linters, 
as proposed before by a Google study~\cite{46576}.
Using this feedback approach, we improved the linter's precision 
and the final ruleset.


\begin{table}[t!]
  \centering
  \small
  \caption{Performance of \toolname{} (v1.1.0). (Validation with Practitioners)}
  \label{tab:final_prac}
  \vspace{-2ex}  
  \begin{tabular}{ p{3.75cm} p{0.5cm} p{0.5cm} p{1cm} p{1cm}} 
    \toprule
    Rule & \#TP & \#FP  & \#Unsure & Precision\\
    \midrule
    Hard-coded secrets & $28$ & $8$ & $3$ & $0.78$ \\
    Use of HTTP without TLS & $32$ & $3$ & $2$ & $0.91$\\
    Suspicious Comments & $16$ & $15$ & $7$ & $0.52$\\
    Use of Weak Crypto. Algo. & $33$ & $3$ & $6$ & $0.92$\\
    Invalid IP Address Binding & $26$ & $8$ & $6$ & $0.77$\\
    Empty Password & $33$ & $3$ & $1$ & $0.92$ \\
    Admin by default & $30$ & $6$ & $6$ & $0.83$\\
    Malicious Dependencies & $25$ & $6$ & $3$ & $0.81$\\
    Weak Password & $32$ & $2$ & $0$ & $0.94$\\
    \midrule
    Total & $255$ & $54$ & $34$ & $0.83$\\
    \bottomrule 
  \end{tabular}
\end{table}

\begin{table}[t!]
  \small
  \centering
  \caption{Precision obtained in different cycles of feedback collection for \toolname{}.}
  \label{tab:infrasecure_prec}
  \vspace{-2ex}  
  \begin{tabular}{p{5cm}p{1.25cm}p{1.25cm}} 
    \toprule
    \textbf{Participants} & \textbf{version} & \textbf{Precision} \\
    \midrule
    Research Team, Owners of OSS Projects, PuppetLabs, Voxpupuli & v0.1.0 &  76\% \\
    Practitioners (cycle 1) & v1.0.0 &  79\%\\
    Practitioners (cycle 2) & v1.1.0 & \finalPrecision \\
    \bottomrule 
  \end{tabular}
  \vspace{-3ex}
\end{table}


\subsection{Discussion \& Limitations}\label{sec:discussion}


This paper reports our approach to improve the ruleset of an IaC security iteratively 
linter in different cycles of feedback collection.
However, the tool can still be improved with more sophisticated 
techniques such as data-flow analysis, which would fulfil the following 
feedback: \textit{In puppet, pre-defining a password as empty does not 
mean it is empty (e.g., \CodeIn{\$ssl\_password    = ''}). Many times 
these variables are changed later. Thus, for each empty password,
\toolname{} verifies if the same variable was changed within the same file.
If it was, then the linter will not raise an issue.}.

In addition, some engineers suggested that usernames 
should be only reported as hard-coded secrets when 
paired with a password/key. For this, we must match the different 
pairs of credentials in a puppet manifest.
To sum up, there are still opportunities 
to improve the precision and recall of \toolname.
We reached out to owners of highly 
active GitHub projects that use Puppet reporting warnings 
detected by \toolname{}. Two owners mentioned that 
since the apps are not in production, they 
did not consider the issues relevant. Even 
after improving the linter to detect the anti-patterns 
correctly, some problems are still not problematic. This happens  
because the linter does not have context regarding 
the software's usage, which will always be a source 
of False Positives. In the future, we will continue to search 
for solutions to make the linter more context-aware
since this is a known problem of linters.

\section{Ethical Standards and Compliance}\label{sec:ethics}

This section discusses compliance with the ACM Policy for Research 
Involving Humans,\footnote{The ACM Policy for Research Involving Humans description is available at \url{https://www.acm.org/publications/policies/research-involving-human-participants-and-subjects} (Accessed \today)} which ensures that the ethical and legal standards 
are met when research has human participants.

\textbf{Informed Consent.} One of the principles is to ensure that 
participants are informed about the fact that they are participating 
in a study. In our study, consent was collected differently for each 
experiment: for the first one, the research team agreed to 
participate in the study; for the OSS maintainers experiment, we used 
the puppet community slack to communicate and discuss the investigation 
with the maintainers; finally, for the practitioners' experiment, we 
asked survey participants if they agreed to participate in our different 
surveys at the beginning of the pre-screening phase. 

\textbf{Data Privacy.} For all experiments, we ensured that the 
participants' private information was protected by not providing 
the participants' personal data (e.g., 
GitHub usernames of the OSS maintainers, prolific participants' 
names, ages, nationalities, etc.) in our replication package. 

\textbf{Spam.} As mentioned in Section~\ref{subsec:maintainers}, 
we carefully organized the issues 
to minimize the amount of messages sent to maintainers~\cite{10.1145/2961111.2962628,10.1145/3379597.3387462}. Security smells of the same kind were all reported in 
a single GitHub issue. In addition, we designed the issues to be 
actionable by providing personalized fix suggestions and 
adding references that document the detected problems.

\textbf{Full Disclosure in Security.} Fully disclosing vulnerabilities on GitHub issues 
allows hackers to exploit unfixed vulnerabilities, creating risks for software users. 
Ideally, vulnerability disclosure should be performed confidentially. Yet, GitHub does not 
provide a feature to report them privately. Therefore, carefully performing full disclosure 
is accepted by OSS maintainers~\cite{fulldisc.2005,askanethcode}. Some OSS projects adopt 
security mailing lists. In those cases, disclosure should be performed through those mailing 
lists. 

\section{Threats to Validity}\label{sec:t2v}

This section presents potential threats to the validity of this study.

\textbf{Internal Validity:} As with any implementation, the scripts that we 
developed to run the tools and collect the metrics 
reported in the paper are potentially not bug-free. However, the scripts and 
outputs are open-source for other researchers and
potential users to check the validity of the results. 

\textbf{Construct Validity:} A potential threat is the manual analysis
of the warnings raised by \slic{} in the Puppet scripts, which can potentially be 
mislabeled. We tried to mitigate this by running a kappa analysis between the two co-authors. 
For the experiments with the OSS maintainers, 
we inferred the validations of the alerts from their comments. Although both co-authors inferred the validations, and a kappa analysis was performed, we risk our inference being incorrect. It is also important to mention 
that even though we made an effort to collect feedback from experienced humans,
their judgement can also not be $100\%$ accurate, which can introduce error  
in the precision values reported.

\textbf{External Validity:} A potential threat to external validity 
is related to the fact that the set of Puppet scripts we have considered 
in this study may not accurately represent the whole set of 
vulnerabilities that can happen during development. We attempt to reduce
the selection bias by gathering a large collection of real, openly 
available (hence, reproducible) Puppet scripts. Another potential threat 
is that we could have missed the latest updates to \slic{}. To
mitigate this risk, we contacted the authors of \slic{} to confirm that 
the version available is true to the most recent one. 
%
\section{Related Work}\label{sec:rw}
%
As IaC has become popular and prevalent, researchers 
have dedicated efforts to improve its quality. Jiang \etal{} conducted
an empirical study on Puppet scripts to gain a deep understanding of the 
characteristics of such scripts and how they evolve over 
time~\cite{7180066}. Bent \etal{} investigated the quality and
maintainability aspects of Puppet scripts~\cite{van2018good}. 
Furthermore, Rahman \etal{} proposed prediction models (based on text
mining) to classify defective IaC scripts~\cite{rahman2018characterizing}. 
Palma \etal{} created a catalog of software metrics for IaC scripts ~\cite{dalla2020towards}. 
In addition, recent work has been developed to detect malicious
packages published on registry maintainers such as npm and ruby
gems~\cite{duan2020measuring}. Building and introducing linters 
earlier in the software development life-cycle shift security 
left and decreases the probability of shipping malicious packages.

There are several linters available for security but only the subject of
this paper, \slic{}, focuses in IaC scripts for Puppet~\cite{8812041}. 
The authors started by demonstrating the linter in the context of Puppet
scripts, and later, the authors reproduced the same study for Chef and 
Ansible and created new tools for those technologies~\cite{10.1145/3408897}. 
A major issues with linters is their lack of precision~\cite{park2016battles,muske2016survey,gauthier2018scalable,landman2017challenges,christakis2016developers,vassallo2020developers}: 
low precision entails low reliability for developers. Previous research 
has shown the impact of this issue on the developers' workflow and stressed 
it is essential to create precise tools; otherwise, the developers 
will not use them~\cite{6606613,7781843,10.1145/1646353.1646374,8622456,8530713,46576}.
As mentioned before, this study aims to gain a better 
understanding of the current capabilities of the only IaC security linter for 
Puppet and shed some light on how to move forward.

%
\section{Conclusions \& Future Work}\label{sec:conclusion}
%
In this study, we observed through a comprehensive study that 
security linters for IaC scripts still need to be improved to 
be adopted by the industry. This paper leverages community expertise 
to address the challenge of improving the precision of such linting 
tools. We focused on precision as it is critically important 
in this domain---false security warnings can be very disruptive. 
More precisely, we interviewed professional developers
of Puppet scripts to collect their feedback on the root causes of
imprecision of the state-of-the-art security linter for Puppet. From
that feedback, we developed a linter adjusting \noRulesSlic\ rules
of an existing linter ruleset and adding \newRules\ new rules. We
conducted a new study with \noProfessionals\ professional developers,
showing an increase in precision from \botPrecision\ to \finalPrecision.
Following the findings of a Google study~\cite{46576}, we show that 
authors of linters can improve their own tools if they focus 
on the users' feedback.
The takeaway messages of this paper are that (i) it is feasible to
tune security linters to produce acceptable precision; and, that 
(ii) involving practitioners in
discussions is an effective way to guide the improvement of those
linters. 

The observations that we made throughout this work pave the
way for the following future work: extend \toolname\ to detect other
security vulnerabilities, integrate the tool with methods for
automated patching, and port \toolname\ to other configuration
management tools.

\section{Using Generative AI to improve the recall of a security linter}
\label{sec:sectionb}


\cleardoublepage

% %%%%%%%%%%%%%%%%%%%%%%%%%%%%%%%%%%%%%%%%%%%%%%%%%%%%%%%%%%%%%%%%%%%%%%
% Dummy Chapter:
% %%%%%%%%%%%%%%%%%%%%%%%%%%%%%%%%%%%%%%%%%%%%%%%%%%%%%%%%%%%%%%%%%%%%%%

% %%%%%%%%%%%%%%%%%%%%%%%%%%%%%%%%%%%%%%%%%%%%%%%%%%%%%%%%%%%%%%%%%%%%%%
% The Introduction:
% %%%%%%%%%%%%%%%%%%%%%%%%%%%%%%%%%%%%%%%%%%%%%%%%%%%%%%%%%%%%%%%%%%%%%%
\fancychapter{SAST Testing and Validation}
\label{cap:chapter}

\textit{Present the chapter content.}

\section{Section A}
\label{sec:sectiona}

\subsection{Subsection A}
\label{subsec:subasectionA}

This would be a citation \cite{dummy}.

The \gls{cop} defines the performance of the machine.
% The first time you use this, the acronym will be written in full with the acronym in parentheses: supernova (SN). At later times it will just print the acronym: SN.

Heat Pump's performance is given by the \gls{cophp}, a \gls{cop} for heat pumps.

\section{Section B}
\label{sec:sectionb}

\subsection{Subsection A}
\label{subsec:subasectionB}


\cleardoublepage

% %%%%%%%%%%%%%%%%%%%%%%%%%%%%%%%%%%%%%%%%%%%%%%%%%%%%%%%%%%%%%%%%%%%%%%
% Dummy Chapter:
% %%%%%%%%%%%%%%%%%%%%%%%%%%%%%%%%%%%%%%%%%%%%%%%%%%%%%%%%%%%%%%%%%%%%%%

% %%%%%%%%%%%%%%%%%%%%%%%%%%%%%%%%%%%%%%%%%%%%%%%%%%%%%%%%%%%%%%%%%%%%%%
% The Introduction:
% %%%%%%%%%%%%%%%%%%%%%%%%%%%%%%%%%%%%%%%%%%%%%%%%%%%%%%%%%%%%%%%%%%%%%%
\fancychapter{Fixing Software Vulnerabilities Potentially Hinders Maintainability}
\label{cap:chapter}

\textit{Present the chapter content.}

\section{Section A}
\label{sec:sectiona}

\subsection{Subsection A}
\label{subsec:subasectionA}

This would be a citation \cite{dummy}.

The \gls{cop} defines the performance of the machine.
% The first time you use this, the acronym will be written in full with the acronym in parentheses: supernova (SN). At later times it will just print the acronym: SN.

Heat Pump's performance is given by the \gls{cophp}, a \gls{cop} for heat pumps.

\cleardoublepage

\input{7.Conclusions/conclusions.tex}
\cleardoublepage
\phantomsection
\addcontentsline{toc}{chapter}{Bibliography}
%% Use with Cite and Natbib
%% check styles in https://en.wikibooks.org/wiki/LaTeX/Bibliography_Management
\bibliographystyle{IEEEtranN}
%\bibliographystyle{IEEEtran}
\bibliography{02.biblio}
%% Use with Biblatex (.bib file is set in Packages) (Not working)
%\printbibliography

\cleardoublepage

\begin{appendices}
	\begin{appendix}
		\pagenumbering{bychapter}
		\input{Appendices/appendixA.tex}     
		\cleardoublepage
	\end{appendix}
\end{appendices}


\end{document}