\section{Section B}
\label{sec:sectionb}

\subsection{Subsection A}
\label{subsec:subasectionB}

The model described can also be represented as

\begin{equation}
\dot{\mathbf{x}}(t) = \mathbf{T}\mathbf{z}(y),\  \mathbf{y}(0) = \mathbf{y}_0,\  z\geq 0 \\
\label{eq:dummyeq1}
\end{equation}

\noindent where

\begin{equation}
\mathbf{A} = \left[ \begin{array}{cc} -(a_{12} + a_{10}) & a_{21} \\ a_{12} & -(a_{21} + a_{20}) \end{array} \right],\ \mathbf{x} = \left[ \begin{array}{c} x_1 \\ x_2 \end{array} \right] \\
\label{eq:dummyeq2}
\end{equation}

Also, using glossaries in the math environment, you can write
\begin{equation}
\gls{A} = \frac{\gls{mflow}\gls{_v}}{\rho u}
\label{eq:dummyeq3}
\end{equation}

Note that \gls{A} is not \gls{a}.


\subsection{Subsection B}
\label{subsec:subbsectionB}

Another example for the notation section: think about \gls{gamma}.
And \gls{gamma}\gls{_p} with a subscript.

\begin{table}[H]
	\centering
	\caption{Dummy Table.}
	\begin{tabular}{|c|c|c|c|} \hline
		\textbf{Vendor Name} 				& \textbf{Short Name}	& \textbf{Commercial Name}	& \textbf{Manufacturer}	\\ \hline \hline
		\multirow{3}{*}{Text in Multiple Row}		&	ABC				&  ABC\textreg				& ABC SA			         \\ \cline{2-4}
		 								&        DEF				&  DEF\textreg				& DEF SA				\\ \cline{2-4}
										&        GHF			&  GHF\textreg				& GHF SA				\\ \hline
		Text in Single Row					&        IJK				& IJK\textreg				& IJK SA				\\ \hline
		Frescos SA						&        LMN			& LMN\textreg				& LMN SA				\\ \hline
		Carros Lda.						&    \multicolumn{3}{|c|}{Text in Multiple Column}							\\ \hline
	\end{tabular}
	\label{tab:dummytable}
\end{table}

